\documentclass[12pt,egregdoesnotlikesansseriftitles]{scrartcl}
%\usepackage[dvipsnames]{xcolor}
%\usepackage{marginnote}
%% fonts:
\usepackage{fontspec}
\defaultfontfeatures{Ligatures=TeX,Numbers={OldStyle,Proportional}}
\setmainfont{XCharter}
\setsansfont{Alegreya Sans}
\newfontfamily\charis{CharisSIL-R.ttf}[
    BoldFont = CharisSIL-B.ttf,
    ItalicFont = CharisSIL-I.ttf,
    BoldItalicFont = CharisSIL-BI.ttf]
\usepackage{amssymb}
\usepackage{enumerate}
\usepackage{fancyhdr}
%\usepackage{authblk}
%% bibliography
\usepackage[backend=biber,
 bibstyle=%authoryear,%
biblatex-sp-unified,
 citestyle=%authoryear-ibid,%
sp-authoryear-comp,
 maxcitenames=3,
 maxbibnames=99]{biblatex}
\setlength{\bibitemsep}{0pt}
\addbibresource{mybib.bib}
\usepackage{leipzig}
\makeglossaries
% The first arg is how you type it in gloss line, e.g. {dub} means you type in \Dub; the second arg is what is printed, e.g. {dub} means small caps dub will appear in pdf; the third arg is what definition shows up in the glossary
\newleipzig{anim}{anim}{animate}
\newleipzig{aug}{aug}{augmented}
\newleipzig{card}{card}{cardinality}
\newleipzig{circ}{circ}{circumstantive}
\newleipzig{cli}{i}{class I}
\newleipzig{clii}{ii}{class II}
\newleipzig{cliii}{iii}{class III}
\newleipzig{cliv}{iv}{class IV}
\newleipzig{clm}{ma}{masculine class}
\newleipzig{clneut}{neut}{neuter class}
\newleipzig{clveg}{veg}{vegetable class}
\newleipzig{conj}{conj}{conjunction}
\newleipzig{ctp}{con}{contemporary tense}
\newleipzig{dub}{dub}{dubitative}
\newleipzig{hum}{hum}{human}
\newleipzig{ignor}{ignor}{ignorative}
\newleipzig{imm}{imm}{immediate}
\newleipzig{impv}{impv}{imperfective}
\newleipzig{indet}{indet}{indeterminate}
\newleipzig{lim}{lim}{limitative}
\newleipzig{min}{min}{minimal}
\newleipzig{nfut}{nfut}{non-future}
\newleipzig{nm}{nm}{noun marker}
\newleipzig{np}{np}{non-past}
\newleipzig{pc}{precon}{precontemporary}
\newleipzig{pcon}{pcon}{past continuous}
\newleipzig{pp}{pp}{past perfective}
%\newleipzig{proh}{proh}{prohibitive}
\newleipzig{rcgn}{rcgn}{recognitional}
\newleipzig{rempst}{rpst}{remote past}
\newleipzig{rep}{rep}{repetitive}
\newleipzig{rr}{rr}{reflexive/reciprocal}
\newleipzig{sel}{sel}{selective enclitic}
\newleipzig{sub}{sub}{subordinate marker}
\newleipzig{ua}{ua}{unit augmented}

\usepackage{multicol}
\usepackage{graphicx}
\usepackage{gb4e}
\usepackage{cgloss}
\let\eachwordone=\charis
%%\usepackage[multiple]{footmisc}
\usepackage[colorlinks,allcolors=blue]{hyperref}
% \usepackage[nomain]{glossaries}
% \usepackage{glossary-inline}

\newcommand{\ofy}{/125} % the n of languages in the survey (i.e., languages we have data for)

% total number of sources surveyed (incl. those without data) = 138

% total number of sources with data = 124

% I didn't include any instances of "personal communication" within these tallies.

\newcommand{\pn}{Pama-Nyungan}
\hyphenation{cli-tics}

\title{Quantification in Australian languages}
\author{Margit Bowler \and\ Ivan Kapitonov}% {\small ({\href{mailto:margitbowler@gmail.com}{margitbowler@gmail.com}} University of Manchester)} \and \vspace{-.5cm}Vanya Kapitonov {\small ({\href{mailto:moving.alpha@gmail.com}{moving.alpha@gmail.com}} University of Melbourne \&\ CoEDL)}}

\begin{document}
\maketitle

\begin{center}
To appear in \textit{Oxford Handbook of Australian Languages}, ed.\ Claire Bowern\\
* DRAFT: Do not cite without permission *
\end{center}

\section{Introduction}

A number of volumes have shed light on the diversity of quantificational systems cross-linguistically (\citealt{keenanpaperno17,keenanpaperno12};  \citealt{qclp08}, \citealt{bachetal95}). However, with the exception of a relatively small number of publications (see especially \citealt{bowler17}, \citealt{bowernzentz12}, \citealt{alpher01},  \citealt{bittnerhale95}, \citealt{evans95}, \citealt{laughren81}), the quantificational systems of Australian languages remain relatively under-studied. This chapter aims to make some progress towards filling this gap. In this chapter, we give a typological overview of quantificational expressions in Australia based on data from 125 languages.
% I changed the \citep thing only because I wasn't sure how to get a semicolon between the K&P volumes and the Matthewson volume... feel free to change back if you know how to do that! -M

We do not assume a theoretical definition of quantifiers in this chapter (i.e., \citealt{heimkratzer98}); rather, we are generally concerned with lexical items that refer to quantities. This includes terms referring to vague quantities (translational equivalents to English \textit{many}, \textit{few}, \textit{several}\ldots), properties of sets (\textit{all}, \textit{some}, \textit{no}\ldots), cardinalities (\textit{one}, \textit{two}, \textit{three}\ldots), Wh-words referring to quantities (\textit{how many}, \textit{how much}), indefinite pronouns (\textit{someone}, \textit{something}\ldots), and terms referring to ``quantities'' of times (or ``cases,'' in \posscitet{lewis75} terminology) (\textit{always}, \textit{sometimes}\ldots). We do not discuss number marking in agreement systems (see Brody this volume), non-pronominal (in)definiteness, or other lexical items that have been theoretically argued to include quantifiers in their semantic denotations, e.g.\ modals, tenses (see Bednall this volume) or degree expressions.

%We show that Australian languages use a variety of morphosyntactic strategies to express quantificational concepts.

\subsection{Our data}

% 49 = number of nPN languages (including isolates)
% 71 = number of PN languages (including isolates)

We aim to have a sample that is genetically and areally balanced as possible. Our sample includes data from 71 Pama-Nyungan languages from 21 subgroups and 49 non-Pama-Nyungan languages from 18 families, as well as 7 language isolates (sometimes categorized within Pama-Nyungan or non-Pama-Nyungan) and two mixed languages. We include language data from all of the Australian states save Tasmania and the Australian Capital Territory.

Our sample of 125 Australian languages  is drawn from 124 published grammars, grammatical sketches, and dictionaries, as well as personal communications with some language experts.  We generally present data as it is given in the original sources; in a small number of  instances, we standardize some interlinear glosses and orthographies.

When we report that a language ``has'' a quantifier, we mean that the sources that we consulted on the language document this quantifier.\footnote{All of the sources we consulted were written in English, and almost all data in our sample appears to have been collected using English as a metalanguage. As a result, we occasionally encountered challenges working with English translations of Australian language data. For instance, we often encountered English glosses containing quantificational expressions that did not occur in the target language data (e.g.\ plural nouns translated into English using \textit{many}).} We restrain from making strong claims about languages lacking certain quantificational
expressions, since there may be gaps in the collected data, particularly in older sources. To make our generalizations as strong as possible, we typically present the frequency of a given quantificational expression as a proportion of the total number of languages in our sample that have it, i.e., for a given expression, we note how many languages have it out of a total of 125.

\section{General morphosyntactic properties of  quantificational expressions}
% MB: I'm commenting this stuff out for now in an attempt to combine the two subsections of this section. We can change it back if you don't like it!
%In this section, we give an overview of the common morphosyntactic properties of Australian quantificational expressions. We discuss the prevalence of particular quantificational expressions in \S\ref{individquantsection}.

%\subsection{Lexical category of quantifier terms}
A frequent morphosyntactic distinction made in the quantifier literature is between D-quantifiers and A-quantifiers \citep{partee95}. %p544
The former (`D' standing for `determiner') build expressions that are arguments (or parts of arguments) of predicates.  Morphosyntactically, D-quantifiers are associated with nouns (e.g. English \textit{every cat, some dogs}). Conversely, A-quantifiers (`A' standing for adverbs, auxiliaries, affixes, argument-structure adjusters, and so on) are used directly to build predicates (cf.\ \citealt{keenan17qu}). Adverbs of quantification (e.g.\ English \textit{usually, seldom}) are a paradigm example of A-quantifiers.

We refer to this distinction to describe some of the morphosyntactic properties of quantificational expressions in Australian languages. We find that basic, set-describing quantifiers (i.e., translational equivalents of English \textit{all, many}, and so on) are frequently realized as nouns. We diagnose a lexical item as a noun by its ability to host case marking and/or trigger agreement marking, as in (\ref{allerg})--(\ref{agrmarking1}),\footnote{The abbreviations used in examples are: \printglossary[style=inline,type=\leipzigtype]} or if it appears within noun phrases.\footnote{Australian languages are often described as lacking adjectives, which are generally indistinguishable from nouns with respect to their morphosyntactic properties (\citealt[67--68]{dixon02}; see also \textbf{XXXX}'s chapter in this volume).} As such, D-quantifiers are relatively widespread in Australian languages.

These nominal quantifiers can typically stand alone as arguments, without any other associated noun, as in (\ref{allerg})--(\ref{agrmarking1}). These quantifiers are also frequently documented in discontinuous NPs, as in (\ref{discconst}) (cf.\ \citealt[51--52]{louagieverstraete16}, who observe that quantifiers are the most frequent type of modifier to occur discontinuously in Australian languages).\footnote{Some of these examples of discontinuous nominal quantifiers may be instances of quantifier float (i.e., stranding of quantifiers by syntactic movement). However, in the absence of syntactic tests showing that examples like (\ref{discconst}) are instances of quantifier float, we remain agnostic as to their source.}

\begin{exe}
  \ex\label{allerg} \textsc{Bardi (nPN: Nyulnyulan)}\hfill \pgcitep{bowern12}{272}\\
  \gll Nyalaboo i-ng-arr-ala-n \textbf{boonyja}-nim.\\
  there \Third-\Pst-\Aug-see-\Rempst{} all-\Erg\\
  \glt `Everyone saw him.'
  \ex \textsc{Warlpiri (PN: Ngumpin-Yapa)} \hfill(\citealt[967]{bowler17})\\
  \gll \textbf{Panu}-ngku=lu karlaja yunkaranyi-ki.\\
  many-\Erg=\Tpl.\Sarg{} dig.\Pst{} honey.ant-\Dat\\
  \glt `Many [people] dug for honey ants.' \label{agrmarking1}
  \ex \textsc{Matngele (nPN: Eastern Daly)} \hfill(\citealt[54]{zandvoort99})\\
  \gll \textbf{Nembiyu} ardiminek \textbf{binya} \textbf{jawk}.\\
  one \First\Min\Sarg.do.\Pst{} fish black.nailfish\\
  \glt `I got one black nailfish.'  \label{discconst}
\end{exe}


In addition to these D-quantifiers, many languages in our survey also have A-quantifiers. These languages express quantificational concepts through verbal modifiers such as free adverbs (\ref{aquant2}), preverbs/coverbs (\ref{aquant3}), and verbal affixes (\ref{aquant1}).
\begin{exe}
  \ex\textsc{Garadjari (PN: Marrngu)} \hfill(\citealt[54]{sands89})\\
  \gll \textbf{wiridjardu}  nga-njari-djinja.\\
  completely  eat-\Cont-\Tpl.\Parg\\
  \glt `He ate them all up.' \label{aquant2}
  \ex \textsc{Warlpiri (PN: Ngumpin-Yapa)} \hfill (\citealt[975]{bowler17})\\
  \gll Karnta=lu \textbf{muku} yanu Nyirrpi-kirra.\\
  woman=\Tpl.\Sarg{} all/completely go.\Pst{} Nyirrpi-\All\\
  
  `All the women went to Nyirrpi.'\label{aquant3}

  \ex \textsc{Mayali (nPN: Gunwinyguan)} \hfill(\citealt[221]{evans95})\\
  \gll Gunj barri-\textbf{bebbe}-yame-ng.\\
  kangaroo \Third.\Aug-\Distr-spear-\Pp\\
  \glt `They each killed a kangaroo.'\\
  ([they]$_{key}$ [kangaroo-spear-bebbeh]$_{share}$, i.e.\ one kangaroo-killing per man) \label{aquant1}
\end{exe}

A small number of quantifiers in our sample ($ n < 10$) are restricted to modifying absolutive arguments. This property is primarily described of A-quantifiers, as in (\ref{quantscope2}) (and (\ref{aquant3}) above). However, \cite{harvey92} describes one D-quantifier in Gaagudju, \textit{geegirr}, that is preferred (but not required) in combination with absolutive arguments (\ref{quantscope3}). (We note that this property does not extend to all of the quantifiers within a given language; for instance, Warlpiri has other (D-)quantifiers that are not restricted to modifying absolutive arguments.)

\begin{exe}
  % \ex  \textsc{Warlpiri (PN: Ngumpin-Yapa)} \hfill(\citealt[15]{bowler17})\\
  % \gll Wati-ngki \textbf{muku} rdilyki-pungu kurlarda-wati.\\
  % man-\textsc{erg} all/completely break.\textsc{pst} spear-several\\
  % \glt `The man broke all the spears.' \label{quantscope1}
  \ex\textsc{Mayali (nPN: Gunwinyguan)} \hfill(\citealt[233]{evans95})\\
  \gll Aban-\textbf{djangged}-bukka-ng.\\
  \First$>$\Tpl-bunch-show-\Pp\\
  \glt `I showed them the whole lot.' \label{quantscope2}
  \ex\textsc{Gaagudju (nPN: isolate)} \hfill(\citealt[307]{harvey92})\\
  \gll ba-'rree-ng-ga=mba \textbf{geegirr} ma'rree-ya=mba \textbf{geegirr}.\\ 
  \Second.\Abs-\First.\Erg-\Fut-take=\Aug{} all \First.\Incl.\Abs-go.\Fut=\Aug{} all\\
  \glt `I will take all of you. We will all go.' \label{quantscope3}
\end{exe}

%has emerged from the last 30 years of research into quantification is
Finally, none of the languages in our survey appear to only use A-quantifiers; we find that A-quantifiers always occur in addition to D-quantifiers. This is  interesting due to the important typological generalization made by e.g.\ \cite{introqnl} that  while languages can lack D-quantifiers, no language has been found to lack A-quantifiers. Our study tentatively suggests that Australian languages  conform to this generalization. However, the nature of our data precludes strong theoretical conclusions about the absence of A-quantifiers in any given language. We believe that Australian languages present an important descriptive lacuna in this area, and could potentially represent typologically unattested quantifier systems.


\section{Semantic findings \label{individquantsection}}

In the following sections, we discuss the prevalence of particular quantificational expressions in the languages in our survey, and review the morphosyntactic strategies that the languages use to encode them.

\subsection{Expressing `many'/`much' \label{manymuchsection}}

Nearly all of the languages in our survey (109\ofy) have a lexical item that contributes a meaning like English \textit{many}, i.e., that the cardinality of a set exceeds some contextual standard.\footnote{Only one language in our survey, the Gooniyandi mother-in-law language, is explicitly described as lacking a word for `many' (\citealt[636]{mcgregor89}).} We frequently find that languages  have more than one lexical item used to express `many,' as demonstrated in (\ref{manypl1}). %(\ref{manypl2}). % in opposition to the popularized view that Australian languages have ``simple'' quantifier systems.

\begin{exe}
  \ex  \textsc{Yugambeh (PN: Ngumpin-Yapa)} \hfill(\citealt{sharpe98}) \label{manypl1}
  \begin{xlist}
    \ex \textit{kamaybu} `lots of,' `plenty,' `beyond four'
    \ex \textit{karal} `more,' `many,' `a lot,' `all,' `plenty'
    \ex \textit{walal}  `many' 
  \end{xlist} 
  % \ex  \textsc{Bardi (nPN: Nyulnyulan)} (\citealt{bowern12})  \label{manypl2}\\
  % \begin{xlist}
  %   \ex \textit{niimana}  `plenty,' `many'\\
  %   \ex \textit{ngarri} `a lot,' `much'\\
  %   \ex \textit{alboorr(oo)} `plenty,' `many'\\
  % \end{xlist} 
\end{exe}

Australian languages typically do not lexically distinguish between quantification over count nouns versus mass nouns, i.e., the distinction between English \textit{many} and \textit{much}. In fact, to the best of our knowledge, there is no published account of the mass/count distinction for \textit{nouns} in any Australian language, and several of our colleagues in personal communication have expressed doubts about relevance of this category for Australian languages altogether. Admittedly, this is as instructive as it is anecdotical. For want of a study of the phenomenon, in this section we talk about countability of the nominal lexemes as if they were English ones. At the same time, notice that the idea about absence if the countability distinction in Australian languages is corroborated by the fact that in all surveyed languages, one lexical item can %therefore
modify both (alleged) count and mass nouns, as in (\ref{manymuch1}).

% MB: I weakened the statement above; I think Warlpiri "panu" can only go with count nouns

\begin{exe}
  \ex \textsc{Biri (PN: Maric}) \hfill(\cite[54]{terrill98}) \label{manymuch1}
  \begin{xlist} 
    \ex \gll  yara    \textbf{dhalgari} mari  wuna-lba-dhana \\
    there  many    men-\Abs{} lie-\Cont-\Pst-\Tpl.\Sarg/\Aarg\\
    \glt `Many men used to live here.'
    \ex \gll \textbf{dhalgari} gamu wara-mba-li gunhami gamu yinda-lbaŋa-la\\
    much  water-\Abs{} be-\Caus-\Pst{} that-\Abs{}  water-\Abs{}  rise-\Cont-\Prs-\Tsg.\Sarg/\Aarg\\
    \glt `Much rain made the river rise.'
  \end{xlist}
\end{exe}

An exception to this is the use of lexical items for `big' to express a quantity meaning akin to English \textit{much}. (At least) 12/109 languages in our sample permit their lexical item for `big' to refer to `a large quantity of [noun].' This almost always occurs in combination with mass nouns, as in (\ref{bigquant1}).\footnote{\citet[37]{louagieverstraete16} assert that in Gooniyandi, prenominal `big' functions as a quantifier, whereas postnominal `big' has an adjectival meaning:
    \begin{exe}
      \ex  \textsc{Gooniyandi (nPN: Bunuban)} \hfill(\citealt{mcgregor90})
      \vspace{-3mm}
      \begin{xlist}
        \begin{multicols}{2}
          \ex \gll \textbf{nyamani} gamba\\
          big water \\
          \glt `a lot of water'% (McGregor 1990: 260) 
          \ex \gll yoowooloo \textbf{nyamani} \\
          man big \\
          \glt `a big man' %(McGregor 1990: 265)
        \end{multicols}
      \end{xlist}
    \end{exe}
} 
(We suspect that the actual number of languages that permit a quantity reading of `big' is significantly higher than this; most language descriptions do not include `big' in their discussion of quantifier systems, as its primary use is not quantificational.) Only one of these 12 languages (Garrwa) is described as permitting `big' to combine with count nouns under a quantity reading, as in (\ref{bigquant2}).\footnote{In Miriwoong, the lexical item \textit{ngerreguwung} `big' can undergo partial reduplication to result in \textit{ngerregungerreguwung} `a very large number'/`very many' (\citealt[43]{kofod78}).}\textsuperscript{,}\footnote{In one language, Murrinh-Patha, `big' can be used to express universal quantificational force. It is capable of both mass and wholistic quantificational readings. Notice the use of the root \textit{ngala} in both the qualifier and quantifier functions:
  %,i.e.\ if the head noun denotes either a mass substance or an entity with understood internal structure (whether collective (herd) or not ( the whole house))
  \vspace{-2mm}
  \begin{exe}
    \ex \textsc{Murrinh-Patha (nPN: Southern Daly)}\hfill (John Mansfield, p.c.)\\
    \gll Me-Ngala mup-ka \textbf{ngala} kanam-ka-wat-nime.\\
    foot-big people-\Top{} big be.\Tsg.\Nfut-\Pauc.\Sarg-frequent-\Pauc.\M\\
    \glt `The whole Big Foot mob come here regularly.'
  \end{exe}
}

\begin{exe}
  \ex \textsc{Wagiman (nPN: Wagiman/Wardaman))} \hfill (\citealt[67]{wilson06})\\
  \gll wahan \textbf{buluman} ga-di-n   ginkin-na.\\
  water   big  \Tsg-come-\Prs{} roar-\Asp\\
  \glt `A lot of rain came roaring here.'\label{bigquant1}
%  \ex \textsc{Garrwa (nPN: Garrwan)} \hfill(\citealt{furby77}) \label{bigquant1}\\
  %\begin{xlist}
 %   \ex \gll \textbf{walgu\v{r}a} wadjili\\
 %   big wild.honey\\
 %   \glt `a lot of wild honey,' `much wild honey' % p. 23
 %   \ex \gll \textbf{walgu\v{r}a}-nanji duŋala-nanji\\
 %   big-\Abl{} hill-\Abl\\
 %   \glt `from the big hill' % p. 33
 % \end{xlist}
 % \ex \textsc{Kalaw Kawaw Ya (PN: Western Torres Strait)} \hfill(\citealt[141]{fo91}) \label{bigquant2}\\
 % \gll Yan burumiya lumiz +war moebaygan nanga burum        \textbf{koeyma}    mathan.\\
 % in.vain pig.\textsc{com} hunt.\textsc{pr.pf} other person.\textsc{erg} when   pig.\textsc{abs}    big.\textsc{adv}    kill.\textsc{pr.pf}\\
 % \glt `He hunted in vain for a pig while the others bagged many.'
  % MB: Vanya, the more I look at this example, the more I think it's not a good counterexample. The word "koeyma" is glossed as "big.ADV," suggesting that it's actually an adverbial modifying the verb.
  \ex \textsc{Garrwa (nPN: Garrwan)}\hfill \pgcitep{mushin12}{80}\\
  \gll Baki kuyu nurr=i waw\char`~, daru-muku yalu-nya kula-ni, wawarra, \textbf{balalanyi}-\textbf{muku} nayi-muku.\\
  and bring we.\Excl.\Nom=\Pst{} {\ } uninitiated.boys-\Pl{} they-\Acc{} south-\Abl{} child big-\Pl{} this-\Pl\\
  \glt `We also brought young boys from the south, child(ren), \textbf{this big mob}.'\label{bigquant2} %(21.3.00.1.DmcD)
\end{exe}

We note  that a small number of languages in our survey have lexical items that can be interpreted as either `many'/`much' or `all'/`every', i.e.\ they are compatible with both existential and universal quantificational force. We discuss these data further in \S\ref{alleverysection}.

\subsection{Expressing `all'/`every' \label{alleverysection}}

Approximately half of the languages in our survey (64\ofy) have at least one strategy for expressing universal quantification over individuals.\footnote{In this section, we primarily discuss expressions that convey collective universal quantification. Overall, relatively few sources in our sample described distributive universal quantifiers like English \textit{each}. For now, we simply note two trends that we observe in translational equivalents of \textit{each}. The first is reduplication; in {\charis{Djambarrpuyŋu}} (\citealt[469]{wilkinson91}) and Miriwoong (\citealt[43]{kofod78}), `each' can be expressed by reduplicating `one,' (e.g.\ Miriwoong \textit{djerrawidjerrawiyang} `each' $<$ \textit{djerrawiyang} `one'). The second trend is the use of  adverbials that invoke notions of spatial distribution  (e.g.\ Warlpiri \textit{jarnku} `each'/`separately'; \citealt{bowler17}). \citet{alpher01} makes similar observations.}  We describe three primary strategies for this purpose, in the order of their frequency: (i) having a unique lexical item with universal force; (ii) having a single lexical item that is compatible with readings of both existential and universal force (i.e., both `many' and `all'/`every'); and (iii)  morphologically deriving `all'/`every' from `many'.\footnote{An additional, uncommon strategy for encoding universal force appears to be the use of morphology encoding something like set closure; this gives rise to an exhaustive interpretation of the plural noun it combines with, resulting in a reading of universal quantification. We find possible set closure suffixes in only 3/64 languages in our sample. We note that the Warlpiri suffix \textit{-patu} in (\ref{patuex1}) is also used to express `several'/`a small number;' however, in examples like (\ref{patuex1}), it can be used to mark set closure regardless of the cardinality of the plural noun.

\begin{exe}
  \ex \textsc{Warlpiri (PN: Ngumpin-Yapa)}\hfill (\citealt[974]{bowler17})\\
  \gll Yapa-\textbf{patu}=ju, pina kulpaja=lu.\\
  person-\textsc{patu}=\Top{} again return.\Pst=\Tpl.\Sarg\\
  \glt `[All] the people, they went back.'  \label{patuex1}
  % \ex \textsc{Martu Wangka (PN: Wati)} (\citealt[158]{marsh92}) \\
  % \gll Palu\textbf{-lyu}.\\
  % 3\textsc{sg}-\textsc{terminative}\\
  % \glt (1) `That's all.'\\
  % (2) `That's the lot.'
  \ex \textsc{Wambaya (nPN: Mirndi)}\hfill (\citealt[80]{nordlinger98})\\
  \gll Yarru irr-aji  alaji-\textbf{rdarra}.\\
  go \Tpl.\Sarg-\Hab.\Pst{} boy.I-\textsc{group(nom)}\\
  \glt `All the boys used to go.' 
\end{exe}}

The majority of these languages ($\sim$45/64)  have unique lexical items with universal force; this is by far the most common of the three strategies. These are primarily free lexical items. However, a very small number of (non-Pama-Nyungan) languages ($\sim$5/64) encode universal quantification  through affixes (as in (\ref{univex5})). We give examples of quantifiers with strictly universal force in (\ref{univex3})--(\ref{univex5}).

\begin{exe}
  \ex  \textsc{Arrernte (PN: Arandic)}\hfill (\citealt[132]{wilkins89})\\
  \gll Alertekwenhe pmere \textbf{ingkirreke} artwe-kenhe, artwe-kenhe pmere.\\
  there place all man-\Poss{} man-\Poss{} place\\
  \glt `That there (pointing to a particular site) was a place for all men, a men's site.' \label{univex3}
  \ex  \textsc{Garadjari (PN: Marrngu)}\hfill (\citealt[48]{sands89}) \\
  \gll \textbf{Djarin}-dja barda-ngka yilba-gu-djinja.\\
  every-\Loc{} sun-\Loc{}   throw-\Fut-\Tpl\\
  \glt `Every day he threw them [the people].' \label{univex4}
  % \ex \textsc{Kija (nPN: Jarrakan)}   (\citealt[39]{kofod96})\\
  % \gll taparu-ny\textbf{-ngarrimpi}    kinya-ny\textbf{-ngarrimi}    kalpa-kalpa-ny\textbf{-ngarrimpi} \\
  % pelican-\textsc{m}-all            heron-\textsc{m}-all        spoonbill-\textsc{m}-all\\
  % `...all the pelicans, all the herons, all the spoonbills...'
  \ex \textsc{Ngalakgan (nPN: Gunwinyguan)}\hfill (\citealt[160]{baker08}) \label{univex5}\\
  \gll  miɻppara-\textbf{kappul}   ŋu-pu-woʔwo      jir-$\varnothing$-ŋowiɲ\\
  child-all \Fsg.\Sarg-\Tpl.\Obj-give.\Pp{} \Fpl.\Sarg-\Third.\Obj-eat.\Pcon\\
  % ai bin gibit ola biganini, [mela] bin dagat na
  \glt `I gave it [food] to all the children, and we ate.'
\end{exe}

The two other primary strategies for expressing universal quantification  involve lexical items for `many'. A small number of languages ($\sim$10/64) have quantifiers that appear to be ambiguous between existential and universal force, as in Kokatha \textit{\charis{muɻga}} in  (\ref{manyallambig1}). %@ Gugada (of Platt) --> Kokatha (more current, after Austlang)
A single lexical item can therefore be interpreted as `many' or `all'/`every', depending on the context.\footnote{In Ngaliwuru, a single lexical item, \textit{mulu}, is glossed as both `all' and `three' (\citealt[77]{boltetal71}). This is the only language in our sample in which we observe this polysemy. However, \cite{platt72} notes that the lexical item \textit{{\charis{maɳgur}}} `all' in Kokatha is historically derived from `three.'} At present, we speculate that these lexical items have an underlying meaning of `many' that can be strengthened to `all'/`every' in some contexts, perhaps pragmatically; however, much further fieldwork is needed to determine how and when this strengthening occurs.\footnote{\cite{bittnerhale95} give a semantic account of a Warlpiri quantifier, \textit{panu}, which can be variably interpreted as either `many' or `all'/`every.' They propose that in its existential strength reading, \textit{panu} is of the same semantic type as common nouns in Warlpiri (i.e., $<$\textit{e,t}$>$). They derive its universal strength reading through a semantic type-shifting operation in which it can be interpreted definitely (i.e., as something of type \textit{e}).} %This gives rise to its reading of universal strength.}

\begin{exe}
  \ex\label{manyallambig1}\textsc{Kokatha (PN: Wati)}\hfill (\citealt[56--65]{platt72})
  \begin{xlist}
    \ex \gll badu ŋurbara \textbf{muɻga} {djiɳɖu galaɭa} njina:djinj.\\
    man  strange  many/all  midday         sit.down\\
    \glt `A lot of strangers sat down at midday.'
    \ex \gll uɭa ambuɖa \textbf{muɻga} ŋur-ŋga\\
    boy   small    many/all      camp-\Loc\\
    \glt `All the boys are at camp.'%\footnote{\cite{platt72} reports that \textit{muɻga} historically only had the meaning `many,' which developed into `all.'}
  \end{xlist}
 \end{exe}
 
 A still smaller number of languages morphologically derive their universal force quantifier from `many'; we observe this strategy in only 6/64 languages in our sample.\footnote{Interestingly, we find evidence for the opposite pattern in  Yir Yoront. In this language, reduplicating the monomorphemic lexical item \textit{moq} `all' yields the (existential force) value judgment quantifier `quite a few':
\vspace{-2mm}
\begin{exe}
  \ex \textsc{Yir Yoront (PN: Paman)}\hfill (\citealt[375]{alpher73})\\
  \textit{\charis moqmor} `quite a few' $<^*$\textit{\charis moqo} `all'
\end{exe}}  Languages accomplish this  through a number of morphological strategies, primarily (i) the addition of a lexical item meaning `only' or `still,' as in (\ref{manyderive1});  and (ii) partial or total reduplication of `many,' as in (\ref{manyderive2}). Interestingly, we note that languages that morphologically derive their universal quantifiers from `many' (as their primary strategy for universal quantification) tend to be non-Pama-Nyungan.
%\footnote{Lesser documented strategies... Burarra}

\begin{exe}
  \ex \textsc{Matngele (nPN: Eastern Daly)}\hfill (\citealt{zandvoort99}) \label{manyderive1}
  \begin{xlist}
    \ex \gll woerreng \textbf{mutjurr} lerr-ma-burrudak-awa\\
    mosquito many bite-\Impf-\Third\Aug.\Sarg.stand.\Pst-\First\Min.\Obj\\
    \glt `Lots of mosquitoes were biting me.' (ex.~353) %(p.\ 54)
    \ex \gll mi ngarru-ma-errerr \textbf{mutjurr}-\textbf{ayu}-rnung\\
    tucker \First\Aug-\Prm-\Incl{} many-only-\Purp\\
    \glt `This tucker belongs to all of us.' (ex.~305)
    \end{xlist}
\ex \textsc{Garrwa (nPN: Garrwan)} \hfill(\citealt[54]{mushin12}) \label{manyderive2}
  \begin{xlist}
    \ex \textit{kaja} `many'
    \ex \textit{kajawaja}  `all,' `every' [lit.\ \textit{many\char`~\textsc{rdp}}]
  \end{xlist}
\end{exe}

Like in \S\ref{manymuchsection}, we find that Australian languages do not lexically distinguish between universal quantification over count nouns versus mass nouns, as in (\ref{univmasscount1}). (We take \textit{gaarra} `salt water' in (\ref{univmasscount3}) to be a mass noun.)
\begin{exe}
  \ex \textsc{Bardi (nPN: Nyulnyulan)} \hfill(\citealt[272, 710]{bowern12}) \label{univmasscount1}
  \begin{xlist}
    \ex %\textbf{Boonyja}gid ambooriny \textbf{boonyja} lagallagal ingarrganyinan barda.\\
    \gll \textbf{Boonyja}=gid ambooriny \textbf{boonyja} lagal\char`~lagal i-nga-rr-ganyi-n-an barda.\\
    all=\textsc{then} people all climb\char`~\Rdp{}  \Third-\Pst-\Aug-climb-\Cont-\Rempst{} away\\
    \glt `Then all the people were climbing up [to get away from the rising water].'\label{univmasscount2}
    % Inyjoordina gaarra \textbf{boonyja}.\\
    \ex \gll I-ny-joordi-na gaarra \textbf{boonyja}.\\
    \Third\M-\Pst-dry.up-\Rempst{} salt.water all\\
    \glt `The sea all dried up.'\label{univmasscount3} %(p.\ 710)
  \end{xlist}
\end{exe}

Furthermore,  Australian languages  do not lexically distinguish between universal quantification  over subparts of a singular count noun versus universal quantification over sets of individuals or mass nouns, as in (\ref{univex0}). (We take \textit{walaalu} `country' to be a singular count noun in (\ref{univex1}), as suggested by the definite gloss.)
\begin{exe}
 \ex \textsc{Gaagudju (nPN: isolate)} \hfill (\citealt[307]{harvey92}) \label{univex0}
  \begin{xlist}
      \ex \gll walaalu $\varnothing$-naana \textbf{geegirr}.\\
    country \Cliv-burn.\Pp{} all\\
    \glt `The country is all burnt.' \label{univex1}
    \ex \gll djirriingi njinggooduwa yaa-bu=mba \textbf{geegirr}.\\
    man woman \Third\Cli-went=\Aug{} all\\
    \glt `The men and women have all gone.' \label{univex2}
  \end{xlist}
\end{exe}

The languages in our sample also generally do not appear to lexically distinguish between universal quantifiers that  combine with morphosyntactically singular nouns versus plural nouns, akin to the 
%lexically distinguish between universal quantification over singular versus plural nouns, i.e.\ the distinction between
English contrast between \textit{every} and \textit{all}. However, there are data from Kunbarlang that suggest the possibility of a morphosyntactically singular (distributive) universal quantifier, as in (\ref{ex:wlgsguq}).
\begin{exe}
  \ex\textsc{Kunbarlang (nPN: Gunwinyguan)}\hfill (Kapitonov, field notes)\\
  \gll \textbf{Na}-\textbf{kudji}\char`~\textbf{kudji} ka-warre.\\
  \Cli-\Rdp\char`~one \Tsg.\Nfut-move.\Np\\
  \glt `Everyone is walking by themselves.' \label{ex:wlgsguq} % sy_170610 
\end{exe}

%this generalization is hard to state definitively due to the fact that person and number agreement marking and plural nominal marking tend to be optional in Australian languages. As a result, it can be difficult to diagnose whether a given quantifier is modifying a singular or a plural noun. For now, we note simply that we have not found any universal force quantifiers that specifically select for either singular or plural nouns. %We show a (lack of) such a contrast in (\ref{}).


\subsection{Expressing `several'/`a small amount'}

Approximately half of the languages in our sample (63\ofy) have a strategy for expressing `several' or `a small amount'.\footnote{We note that related quantifiers like English \textit{few} have an associated value judgment that the cardinality of the set they quantify over is below some contextually established expectation (\citealt{keenan17qu}). For the purpose of this chapter, we do not distinguish between quantificational expressions that do or do not have this value judgment. Overall, very few of our sources describe such a value judgment.} We describe four primary strategies for expressing `several': (i) having a lexical item that uniquely encodes `several'; (ii) having a lexical item that is polysemous between expressions of cardinality (e.g.\ \textit{two, three}) and `several'; (iii) morphologically deriving `several' from expressions of cardinality, typically through reduplication; and (iv) using a lexical item for `small' to express `a small amount.'

Of the languages in our sample with a strategy for expressing `several,' almost two thirds of them (41/63) have a unique lexical item with this meaning.\footnote{For the purpose of this count, we attempted to include only lexical items for `several' that are not explicitly described as being synchronically polysemous with/morphologically derived from numerals. However, due to gaps in the descriptions of these languages, we suspect that a number of the lexical items included in this tally are in fact related to numerals.} We give examples of two such lexical items in (\ref{warlpiriseveral1})--(\ref{awabakalseveral1}).
%\pagebreak
\begin{exe}
  \ex \textsc{Warlpiri (PN: Ngumpin-Yapa)} \hfill(\citealt[970]{bowler17})\\
  \gll Napaljarri-rli karlaja \textbf{wirrkardu}.\\
  Napaljarri-\Erg{} dig.\Pst{} few\\
  \glt `Napaljarri dug few [honey ants].'\thinspace\footnote{Historically, \textit{wirrkardu} was used to mean `three' as well as `several.'} \label{warlpiriseveral1}
  \ex \textsc{Awabakal (PN: Yuin-Kuric)} \hfill\pgcitep{lissarrague06}{ex.~176}\\
  \gll \textbf{waraya}   kuri\\
  few:\textsc{abs} men:\textsc{abs}\\
  \glt `few men'\label{awabakalseveral1}
  % \textsc{Bardi (nPN: Nyulnyulan)} (\citealt[271]{bowern12})
  % \gll \textbf{Jalboorr} a-n-a=ngay, joo a-n-ay-a=rr arang.\\
  % a.little 2-\textsc{tr}-give=1\textsc{m.do} 2\textsc{min} 2-\textsc{tr}-take-\textsc{fut}=3\textsc{a.do} others\\
  % \glt `Give me a little, and you take the rest.' 
  % \footnote{\cite{bowern12} glosses \textit{jalboorr} as `few'/`a little'; however, we could not find any examples of it combining with count nouns.} \label{bardiseveral1}
  % \ex Urningangk: wurrinyakburriny \\
\end{exe}

A smaller number of languages ($\sim$15/63) have lexical items that are polysemous between readings of cardinality (i.e., numerals) and vague readings of small quantity.  We find that the numerals used in these expressions range from one to four, as in (\ref{fewex0})--(\ref{fewex3}). `One' has a vague use in only one language in our sample, Mangarayi; \citet[93]{merlan89} notes that \textit{wumbawa} `one' can have a vague interpretation only in combination with inanimate nouns in Mangarayi.\footnote{\citet[143]{bowernzentz12} also describe Warlmanpa as permitting a vague interpretation of `one.' However, we do not currently include Warlmanpa in our language sample.} (Since Australian counting systems have already been discussed at length in \cite{bowernzentz12}, we refer readers to their paper for discussion specifically on numerals. The findings of our study generally accord with their conclusions; for instance, they describe similar vague uses of numerals in their data.)
% MB: They say that no language has a vague use of the word for "two," though! 

%\footnote{\citealt[142-144]{bowernzentz12} describe similar data in some of the languages in their sample.}

\begin{exe}
\ex \textsc{Mangarayi (nPN)}\hfill (\citealt[93]{merlan89})\\
\textit{wumbawa} `one,' `a small number,' `few' [only with inanimates] \label{fewex0}
  \ex \textsc{Djinang (PN: Yolngu)} \hfill(\citealt[9]{waters83})\\
  \textit{bilawili} `two,' `a few' \label{fewex1}
  % \ex \textsc{Alyawarra (PN: Arandic)} yallop 1977: 159\\
  % \textit{irrpitja} `three,' `a few'
  \ex \textsc{Gooniyandi (nPN: Bunaban)} \hfill (\citealt[149]{mcgregor90}) \label{fewex2}\\
  \textit{ngarloodoo} `three,' `a few'
  \ex \textsc{Kuuku Ya'u (PN: Paman)}\hfill (\citealt[27,82]{thompson88})\\
  \textit{mangku} `four,' `a few' \label{fewex3}
\end{exe}

A still smaller number of languages ($<$10/63) morphologically derive their lexical item for `several' from numerals. This morphological derivation is typically accomplished through reduplication, as in (\ref{severalredup1}). Conversely, \citet[51]{alpher73} argues that in Yir Yoront, \textit{\charis{wapayər}} `three' is morphologically derived from \textit{wap} `few,' `some.'


\begin{exe}
    \ex \textsc{Djabugay (PN: Paman)} \hfill (\citealt[87]{patz91}) \label{severalredup1}
  \begin{xlist}
    \ex \textit{mulu} `two'
    \ex \textit{mulumulu} `a few' [lit.\ \textit{two\char`~\textsc{rdp}}]
  \end{xlist}
\end{exe}

Finally, $\sim$5/63 languages in our sample use a lexical item meaning `small' as a primary strategy for indicating a small amount. This parallels the use of `big' to express `much', as described in \S\ref{manymuchsection}. We find no examples of `small' being used in combination with count nouns to express `several'; rather, `small' is used to refer to small amounts of a mass quantity, e.g.\ \textit{kikakkin} `meat' in (\ref{ex:lil}).

\begin{exe}
  \ex\label{ex:lil} \textsc{Kunbarlang (nPN: Gunwinyguan)} \hfill \citep[276, 133]{ikthesis}
  \begin{xlist}
    \ex \gll Kikka ngorro ka-rninganj \textbf{ki}-\textbf{wanjak}, ngadda-karrmeng.\\
    she \Dem.\Med.\Cliv{} \Tsg.\Real-sit.\Pst{} \Clii-little \Fpl.\Excl.\Real-get.\Pst\\
    \glt `She was little [when] we got her.' %%\trailingcitation{[20060620IB03/29:17--22]}
    \ex \gll Kadda-djarrang \textbf{na}-\textbf{wanjak} nayi kikakkin.\\
    \Tpl.\Nfut-eat.\Pst{} \Cli-small \Nm.\Cli{} meat\\
    \glt `They ate a little bit of the meat [but didn't finish it all].' %[ik160802-000/52:15--26]
  \end{xlist}
%  \ex \textsc{Kunwinjku (nPN: Gunwinyguan)} (\citealt[76,116]{carroll76})\\
%  \begin{xlist}
%    \ex \\
%    \gll korroko     bu    ngaye        nga-\textbf{yahwurd}-ni. \\
% long.ago    when    I        1\textsc{sg}-small-was \\
% \glt    `Long ago when I was small.'
%    \ex \gll ku-bo-\textbf{yahwurd} \\
%    \textsc{nc?}-water-small\\
%    \glt `a small quantity of water' 
%  \end{xlist}
  % \ex \textsc{Mengerrdji (nPN: Giimbiyu)} (\citealt[70]{birch06})\\
  % \textit{\charis ngelebenb} `small,' `few'
\end{exe}

Interestingly, we find that expressions for `several' tend to co-occur primarily with count nouns and do not co-occur with mass nouns (unlike the behavior of lexical items for `many'/`much' in \S\ref{manymuchsection} and `all' in \S\ref{alleverysection}). In the absence of significant cross-linguistic negative data, we are unable to make a strong claim with respect to this point. However, we tentatively observe that this generalization appears to hold.


We also observe that there are clear historical links between numerals and expressions for `several' in many of the languages in our sample. This includes synchronic polysemies (as in (\ref{fewex0})--(\ref{fewex3})) as well as historical relationships between `several' and numerals (as in Warlpiri \textit{wirrkardu} (\ref{warlpiriseveral1})). This is documented across both Pama-Nyungan and non-Pama-Nyungan languages, suggesting that it is relatively widespread across Australia.

\subsection{Expressing partitive `some'}
\label{sec:some}
In this section we address translational equivalents for the English partitive quantifier \textit{some}. This lexical item is used to denote a proportion of a total, as in the English expression \textit{Some (of the) cats are black}.\footnote{In English, partitive \textit{some} is homophonous with an existential/indefinite lexical item \textit{some}, which we do not discuss. (This latter \textit{some} occurs in existential expressions like \textit{\textbf{Some} bananas are on the table}, and can be phonologically reduced to [{\charis sm̩]}.) If our source on a given lexical item did not comment on its semantics, it was often not possible to tell with certainty whether we were dealing with a partitive or existential/indefinite `some.' The decision was hard to make in many such cases. Since definiteness is not typically overtly marked in Australian languages, we normally gave these borderline cases the benefit of the doubt and counted them as partitives. %These two lexical items further contrast in that existential \textit{some} can be phonologically reduced to [{\charis sm̩}], whereas partitive \textit{some} cannot.

We would  like to underscore that we are not suggesting an Anglocentric point of view where all lexical items for \textit{some} are necessarily at risk of such partitive/existential ambiguity; rather, our concern stems from the fact that our sources are all in English. Whenever the English translation/gloss was all the information we had on a given lexical item, this uncertainty arose due to the ambiguity in English.} %@ a reference is needed for the definiteness claim in the fn.} Descriptions of 31 language in our sample report a strategy to express the meaning `some (of the \textit{n}s)'.

Approximately a quarter ($\sim$35/125) of the languages in our sample are described as having a quantificational expression akin to partitive \textit{some}. These languages use two main strategies for expressing \textit{some}: (i) having a dedicated lexical item to express partitive \textit{some}, and (ii) having a single lexical item that is polysemous between `some' and `other' or `different'. Half of these languages (\textit{n} = 18) use a dedicated lexical item to express partitive `some', as in (\ref{ex:specsmawa})--(\ref{ex:specsmwlg}).

\begin{exe}
  \ex\label{ex:specsmawa} \textsc{Awabakal (PN: Yuin-Kuri)}\hfill \pgcitep{lissarrague06}{ex.~177}\\
  \gll Anti=pu \textbf{winta} kuri.\\
  here=\Excl{} some.\Abs{} men.\Abs\\
  \glt `Some of the men are here.' % the pdf is quite fucked up and the hard copy seems missing from the melbourne lib, but I thought it'd be good to have this one language for some diversity.
  \ex\label{ex:specsmwlg} \textsc{Kunbarlang (nPN: Gunwinyguan)}\hfill (Kapitonov, field notes)\\
  {\gll Ngunda ki-kala ngob nayi barbung la \textbf{na}-\textbf{yika}\\
  not \Tsg.\Irr.\Pst-get.\Irr.\Pst{} all \Nm.\Cli{} fish \Conj{} \Cli-some\\}
  \sn {ka-(rnak)-kalng.\\
  \Tsg.\Nfut-\Lim-get.\Pst
  \glt `S/he didn't get all the fish, but only got some.'} % sy_160701
\end{exe}

Less than half of these languages (\textit{n} = 14) use a lexical item that can also express `other'/`different', %(or, less frequently, `different'),
as in (\ref{ex:smotherbur})--(\ref{ex:smothernyan}). When `other' is used as a translational equivalent of partitive `some,' it can be repeated more than once within the expression, as in (\ref{ex:smotherbur3}) and (\ref{ex:smothernyan}).
\begin{exe}
  \ex\label{ex:smotherbur} \textsc{Burarra (nPN: Maningrida)}\hfill (\citealt{green87})
  \begin{xlist}
    \ex {\gll Abirri-ny\textbf{=yerranga} marnnga jiny-bunggiya-$\varnothing$ jiny-yorkiya-$\varnothing$\\
    \Third\Ua-\F=other sun \Third\Min-fall-\Ctp{} \Third\Min-do.always-\Ctp{}\\}
    \sn {abirri-ny-bamu-na.\\
    \Third\Ua-\F-go.along-\Pc
    \glt `The other two women went (to where) the sun sets.' \label{ex:smotherbur2}}
    \ex \gll an-\textbf{nerranga} an-mola  rrapa  an-\textbf{nerranga}  an-bachirra.\\
    \Third.\Min-other \Third\Min-good and \Third\Min-other \Third\Min-wild\\
    \glt `Some are friendly and some are the angry kind.' \label{ex:smotherbur3}
  \end{xlist}
  \ex\label{ex:smothernyan} \textsc{Nyangumarta (PN: Marngu)}\hfill \pgcitep{sharp04}{258}\\
  \gll mungka wupartu mayi-rrangu kurrngal \textbf{jinta} juri \textbf{jinta} kari.\\
  tree  small vegetable.food-\Pl{} many other sweet other bitter\\
  \glt `The small tree/bush has lots of fruit [pilirta], some are sweet and some are sour.'
\end{exe}

 We speculate that the partitive reading of these expressions could arise from the presuppositions associated with `other'. In an English assertion like \textit{Other boys ran}, there seems to be a presupposition that some boys in the discourse context did not run.\footnote{We note that the acceptability of this sentence is degraded when it occurs outside of a larger discourse context.  Our goal is not to give a semantics of English \textit{other}; however, we note simply that English \textit{other} could be subject to some discourse anaphoric requirement that the lexical items for `other' in Australian languages are not necessarily subject to.} This in turn leads to a partitive reading of the predicate in which it is true when evaluated against some of the individuals in the discourse context, and false when evaluated against others.

% MB: I'm having a really hard time discussing the presuppositions associated with "other" here. Please feel free to change this if it's not clear! --- IK: I think it's a nice discussion!

% Iara says: Morzycki's book (p. 74) talks about "different" as a "sentence external reading"

 %@ Russian has this word inoj, which also means `other'/`different' and also has the partitive-some reading. It contrasts with drugoj `other' in that it is less dependent on the context --- so I wonder with inoj it is a presupposition indeed. But then there are so many fine details; for one, for the partitive reading it seems best in the sentence-initial position (i.e. when it is in the object, the object needs to be preposed).

Theories of English \textit{some} typically attribute its partitive reading to pragmatic competition with the universal force quantifiers \textit{all/every}. This competition results in a scalar implicature associated with \textit{some} that leads to the reading `some but not all' (\citealt{horn72}). This scalar implicature is the most reliable characteristic of partitive `some' (as opposed to the indefinite `some'); however, this implicature is a fine semantic judgment that requires careful testing in context. The presence of this implicature is confirmed explicitly  for lexical items in only a handful of languages in our sample (n $<$ 10), as in e.g.\ the Kunbarlang  example in (\ref{ex:specsmwlg}). % looks like pitjantjatjara, umpila, murrinhpatha, kunbarlang, and judging from exx probably also burarra, nyangumarta, maybe awabakal
% IK: So there are a few cells where there is a "no" for the L having a `some' expression, but it just looks to me like some Nick bullshit. I.e. he didn't see one and concluded there wasn't one. That's clearly so for the two Giimbiyu Ls (and then for the third one, mengerrdji, he says "yes", apparently b/c there's a little table that has a word for `other' !!!), and I believe I checked the garadjari and umbugarla as well.
% MB: Blegh, thank you for catching this!

We presently hypothesize that when partitive `some' is expressed by `other', this partitive meaning could be encoded as a presupposition (as described above), rather than a scalar implicature.  However, understanding the semantics of `some' and `other' in these languages requires much further research.

%(This predicts e.g.\ that the partitive meaning of `some'/`other' should not be able to be cancelled, unlike the partitive scalar implicature associated with English \textit{some} (cite).) --- NICE! But probably ok to keep out of the chapter, yep.

\subsection{Constituent (nominal) negation}
\label{sec:neg}
%expressions indicating that the intersection of the sets of relevant individuals is empty (English: \textit{No dogs ran}).
For a complete survey of negation in Australian languages, we refer the reader to Phillips (this volume). In this section, we focus solely on expressions of constituent (nominal) negation, e.g.\ English \textit{no dogs}. We find that approximately two thirds of the languages in our sample (84\ofy) have a strategy for uniquely or primarily expressing constituent negation.  We identify two primary morphosyntactic  strategies for this purpose: (i) using free, uninflected lexical items, and (ii) using privative nominal suffixes. (Some languages use both strategies.)

Free lexical items are the most common strategy that languages use to express constituent negation; they are described in 51 languages in our sample. Pama-Nyungan and non-Pama-Nyungan languages are represented roughly equally among these 51 languages. We find examples of negative particles occurring prenominally (\ref{negex1}) as well as postnominally (\ref{negex2})--(\ref{negex3}).

\begin{exe}
  \ex \textsc{Garrwa (nPN: Garrwan)} \hfill (\citealt[37]{furby77}) \\
  \gll \textbf{migu-yadji}    mama-nji    walgura-$\varnothing$    ŋawamba    bayagad̩a-$\varnothing$\\
  nothing  food-\Refr{} big-\Nom{} only  small-\Nom\\
  \glt `There are no big (watermelons) to eat---only small ones.' \label{negex1}

  \ex \textsc{Matngele (nPN: Eastern Daly)} \hfill (\citealt[102]{zandvoort99})\\
  \gll Yim \textbf{dakayu} jawungu ngutjyende-ma.\\
  fire \Neg{} today morning-\Prm\\
  \glt `We had no fire this morning.' \label{negex2} %(p.\ 102) 

  \ex \textsc{Warrongo (PN: Maric)} \hfill (\citealt[660]{tsunoda11})\\
  \gll Banggorro-$\varnothing$ \textbf{nyawa}.\\
  freshwater.turtle-\Nom{} \Neg\\
  \glt `There is no turtle (meat).' \label{negex3}
\end{exe}
 

33 languages in our sample use privative nominal suffixes to express constituent negation, as in (\ref{privexist1})--(\ref{privexist3}). Privatives typically express \textit{without N} or \textit{lacking N}; as a result, these suffixes often occur in expressions describing (a lack of) possession, as in (\ref{privexist2})--(\ref{privexist3}).\footnote{Since property concepts like \textit{tall, short}, etc.\ are typically encoded as nouns in Australian languages, privative suffixes can sometimes also be used to express the absence of a given property, as in (\ref{privproperty1}).
\vspace{-2mm}
\begin{exe}
  \ex \textsc{Warlpiri (PN: Ngumpin-Yapa)} \hfill (\citealt{wdp})\\
  \gll \textit{Wita}, ngula=ji yangka wiri\textbf{-wangu}.\\
  small that=\Top{} that big-\Priv\\
  \glt `\textit{Wita} [small] is something that is not big.'\label{privproperty1}
\end{exe}
} Interestingly, we find that privative suffixes are primarily a feature of Pama-Nyungan languages. Only a small number of non-Pama-Nyungan languages in our sample (n $<$ 10) are described as having privative suffixes.
%Only four non-Pama-Nyungan languages in our sample (Alawa, Miriwoong, Wagiman, and Wardaman) are described as having privative suffixes.

\begin{exe}
  \ex \textsc{Nhanda (PN: Kartu)} \hfill (\citealt[64]{blevins01}) \label{privexist1}\\
  \gll wilu-nggu apa-\textbf{nyida}.\\
  river-\Loc{} water-\Priv\\
  \glt `There's no water in the river.'
  % \ex \textsc{Arabana-Wangkangurru (PN: Karnic)} \hfill (\citealt[237]{hercus94})\\
  % \gll Antha kadnhaardi-\textbf{padni}.\\
  % I money-\textsc{priv}\\
  % \glt `I haven't got any money.' (lit.\ `I have no money.') \label{privexist2}
  \ex\label{privexist2}\textsc{Bilinarra (PN: Ngumpin)} \hfill (\citealt[122]{nordlinger90})\\
  \gll tarnku-\textbf{mulung}-pa-ja ya-ni Yarralin-jirri.\\
  tucker-\Priv-\#-\Fdu.\Excl.\Sarg{} go-\Pst{} Yarralin-\All\\
  \glt `We had no tucker (so) we went to Yarralin.'
  \ex \textsc{Wagiman (nPN: Wagiman/Wardaman)} \hfill (\citealt[150]{wilson06})\\
  \gll ga'an dyilimakun warren-\textbf{ne'en}\\ %[GHY 831]
  that woman child-\Priv\\
  \glt `That woman has no children.' \label{privexist3}
\end{exe}

Finally, we note that a few languages in our sample use specific case frames in constituent negation constructions. For example, in Alawa (\ref{alawanegex1}), the negative particle \textit{madi} co-occurs with genitive case marking on the relevant noun. Similar data is described in Wambaya (nPN: Mirndi) with dative case marking (\citealt[204]{nordlinger98}). Languages that use these morphosyntactically complex strategies are primarily non-Pama-Nyungan; however, we note that these strategies are uncommon overall (described in fewer than five languages).
 
 %in Wambaya (\ref{wambayanegex1}), the negative lexical item \textit{guyalinya}  typically co-occurs with dative case marking on the relevant noun.

\begin{exe}
  % \ex \textsc{Wambaya (nPN: Mirndi)} \hfill (\citealt[204]{nordlinger98}) \\
  % \gll \textbf{Guyalinya}  ngawurniji manganymi\textbf{-nka}.\\
  % lacking.II(\textsc{nom}) 1\textsc{sg.nom}  tucker.III-\textsc{dat}\\
  % \glt `I've got no tucker.' \label{wambayanegex1}
  % \ex \textsc{Burarra (nPN: Burarran)} (\citealt[19]{green87})\\
  % \gll \textbf{gala}     barrwa     mun\textbf{-nga}         manggo.\\
  % \textsc{neg}     again         3M\textsc{mun}-\textsc{indet}.thing    mango \\
  % \glt `There are no more mangoes.'
  % \ex \textsc{Muruwari (PN: Isolate)} \hfill (\citealt[74]{oates88})\\
  % \gll \textbf{wala} mathan\textbf{-pira}\\
  % \textsc{neg} limb-having.\\
  % \glt `There are no sticks.'% (p.\ 74)
  \ex \textsc{Alawa (nPN: Mirndi)} \hfill (\citealt[63]{sharpe72})\\
  \gll nida \textbf{madi} ŋuku-\textbf{yi}\\
  this  no  water-\Gen\\
  \glt `There is no water here.' \label{alawanegex1}
\end{exe}
 
\subsection{Indefinite pronouns}
\label{sec:indefs}
Indefinite pronouns (e.g.\ \textit{someone}, \textit{nothing}, \textit{anywhere}, \textit{whoever}) are often directly associated with existential and universal quantifiers in semantic analyses, and are a prime tool for the study of scope ambiguities and quantifier interaction. However, we do not find much dedicated discussion of these expressions in current descriptive literature on Australian languages. Translational equivalents of English indefinite pronouns  are described in approximately one third (43\ofy) of the languages in our sample.\footnote{Ignoratives (e.g.\ English \textit{whatchamacallit}) are also described in a significant number of languages in our sample; like indefinite pronouns, ignoratives have to do with the existence of an individual without providing the identity of that individual. They are used as genuine hesitation markers, but also very frequently as a speech strategy to avoid direct naming of people or objects, which can be chosen for various reasons, including stylistic.}

For the purpose of this overview, we considered all items that were translated with the English indefinites; however, further classification was usually impossible. For instance, the available data were usually insufficient to distinguish between existential and free choice indefinites, for which the primary difference would be that of scope (i.e.\ the English \textit{someone} vs.\ \textit{anyone}; however, see \S\ref{sec:qfrscope}).\footnote{We do not find any robustly grammaticalized distinctions between specific and non-specific indefinite pronouns (as in e.g.\ the lexical distinction in Russian between \textit{koe-kto} `specific someone who the speaker knows but does not reveal to the addressee' and \textit{kto-nibud'} `non-specific someone [appropriate in conditionals and modal contexts]'). However, very few sources in our sample discuss the kinds of contexts that would uniquely license specific versus non-specific indefinites. Due to this lack of data, we do not claim that Australian languages lack this grammatical distinction; we note simply that we do not find strong cross-linguistic evidence for it.}

In the vast majority of these languages (36/43), the indefinite pronouns are related to the interrogatives (i.e., Wh-words). These languages can be further divided into two groups:
\begin{enumerate}[(i)]
\item Languages in which the indefinite pronouns are identical in form to Wh-words, i.e.\ one form is ambiguous between interrogative and indefinite readings (25/36 languages) %(\ref{ex:blnidf})
\item Languages in which a morphological operation (either optional or obligatory) derives indefinites from Wh-words (at least 11/36 languages)
\end{enumerate}

A lexical item that is ambiguous between the interrogative and the indefinite readings is exemplified in (\ref{ex:blnidf}).

\begin{exe}
  % \ex\label{ex:gniidf} \textsc{Gooniyandi (nPN: Bunaban)} (\citealt[147]{mcgregor90})\\
  % \gll \textbf{ngoonyi}-yidda wardginggiri.\\
  % which-\All{} you.go\\
  % \glt (1)  `Where are you going?' \\
  % (2) `You're going somewhere.'
  \ex\label{ex:blnidf} \textsc{Bilinarra (PN: Ngumpin-Yapa)}\hfill \pgcitep{nordlinger90}{37}\\
  \gll \textbf{Ngantu}-rlu-nga pa-ni.\\
  who-\Erg-\Dub{} hit-\Pst{}\\
  \glt (1) `Who hit him?'\\
  (2) `Maybe someone hit him.'
\end{exe}

This frequent functional duality of the same form is the subject of a (smaller scale) typological study by \citet{mushin95}. She suggests that their epistemological contribution is the basis for such functional development, and coins the term \textit{epistememe} for the forms that serve as ontological categorization of discourse referents and that may take on interrogative, indefinite, hesitation and complementizing functions. We further find that the identical forms will often have different distributional tendencies, e.g.\ when used as interrogatives these pronouns will appear clause-initially (\ref{ex:bkwidf1}), but enjoy more freedom of placement when used as indefinites (\ref{ex:bkwidf2}).
\begin{exe}
  \ex\label{ex:bkwidf} \textsc{Bininj Kun-wok (nPN: Gunwinyguan)}\hfill \pgcitep{evans03}{280--1}
  \begin{xlist}
    \ex\gll \textbf{Njale} bene-boken kabene-h-na-n?\\
    what \Third.\Ua-two \Third.\Ua-\Imm-see-\Np\\
    \glt `What are they two looking at?' \label{ex:bkwidf1} % ex 7.79a
    \ex\gll bu \textbf{njale} ngarri-ma-ng\ldots\\
    \Sub{} what \First.\Aug-get-\Np\\
    \glt `and if we get something\ldots' \label{ex:bkwidf2}% ex 7.90
  \end{xlist}
\end{exe}

We find a range of morphological strategies that are used (either optionally or obligatorily) to derive the indefinites from the interrogatives. These strategies include using indefinite, ignorative, or dubitative affixes/particles (\ref{ex:idftime}) and   reduplication (\ref{ex:idfrdp}). %% \footnote{Interestingly, Bardi (nPN: Nyulnyulan) can express the indefinite pronoun `something' using a compound of `who' and `nose': \textit{angginimal} `something' [lit.\ \textit{anggaba} `who' + \textit{niimal} `nose'] (\citealt[321]{bowern12}).} IK: Taking this out as per Claire's comment. The what-erg-indef seems pretty plausible :)
\begin{exe}
  \ex\label{ex:idftime} \textsc{Djambarrpuyŋu (PN: Yolŋu)}\hfill \pgcitep{wilkinson91}{393}\\
  \gll %nhämunha/nhämuny - ‘How many’ (stem)
  Ga djäma nhe dhu ga-a yindi nhe dhu ga djäma \textbf{ŋula} \textbf{nhämunha} dhuŋgarra ŋurraka$+$m\\
  and work you \Fut{} \Impv-\First{} big you \Fut{} \Impv-\First{} work \Indf2 how.many year throw$+$\First\\
  \glt `And you are working, you are working (on something) big, lasting for an indefinite amount of time.'
  \ex\label{ex:idfrdp} \textsc{Arabana-Wangkangurru (PN: Karnic)}\hfill \pgcitep{hercus94}{129}\\
  \gll \textbf{Thiyara}\char`~\textbf{thiyara} yuka-ka \textbf{minha}\char`~\textbf{minha} mapi-rnda, partyarna ngawi-lhiku waya-rnda.\\
  \textsc{rdp}\char`~which.way go-\Pst{} \textsc{rdp}\char`~what do-\Prs{} all hear-\Purp{} wish-\Prs\\
  \glt `Wherever he went and whatever he did, I want to hear it all.' %(p.\ 129)
\end{exe}
  % \ex  \textsc{Ngiyambaa (PN: Central NSW)} (\citealt[271]{donaldson80}) \\
  % \gll \textbf{ŋa:ndi-ŋa:ndi-ga:} manabi-nji.\\
  % who\char`~who+ \textsc{abs}-\textsc{ignor} hunt-\textsc{past}\\
  % \glt `Whoever went hunting, I don't know.' 
  % \ex  \textsc{Tiwi (Isolate)} (\citealt[57]{osborne74})\\
  % \begin{xlist}
  %   \ex \charis{\textbf{kuwani} jilkəɹimi?}\\
  %   `Who did it?'
  %   \ex \textbf{aramu-kuwarni}.\\
  %   `Someone or other.'
  % \end{xlist}

Besides this major strategy of forming indefinites from interrogatives, there are two other minor trends. One is using generic nouns or classifiers (as in (\ref{indetgeneric1})--(\ref{indetclass1})) to fulfil the function of indefinite pronouns. We find this strategy in 6/43 languages in our sample. (Languages may use these strategies in addition to deriving indefinites from Wh-words.)
\begin{exe}
  \ex \textsc{Warlpiri (PN: Ngumpin-Yapa)} \hfill \citep{wdp}\\
  \gll \textbf{Yapa} ka ya-ni-rni.\\
  person \Aux.\Prs{} go-\Np-hither\\
  \glt `Someone is coming.' \label{indetgeneric1}
  \ex\label{indetclass1} \textsc{Burarra (nPN: Maningrida)}\hfill \pgcitep{green87}{9}\\
  {\gll an=gata    \textbf{ana}=\textbf{nga}            joborr    gu-rrumu-rra\\
  \Third\Min=that.\Rcgn{}   \Third\Min\Hum=\Indet{}   law     \Third\Min$>$\Third\Min-break-\Pc\\}
  \sn {\gll abu-bu-na aburr-workiya-na.\\
  \Third\Aug$>$\Third\Min-hit-\Pc{} \Third\Aug-do:always-\Pc\\
  \glt `When someone breaks the law, they always hit him.' [translation ours---MB\&IK]}
\end{exe}
  
The other option is for a language to have dedicated lexical items for indefinite pronouns. We have identified 8/43 languages that have such items.
\begin{exe}
  \ex \textsc{Kalkatungu (PN: Kalkatungic)}\hfill \pgcitep{blake79}{104--5}
  \begin{xlist}
    \ex \textit{\charis n̪ani} `who'; \textit{\charis n̪aka} `what'
    \ex ``The interrogatives are not used as indefinites\ldots\ \textit{\charis ŋarpa} is the indefinite `some creature'\ldots\ \textit{\charis min̪aŋara} is `something'\thinspace''
  \end{xlist}
\end{exe}

Expressions with translational equivalents of negative indefinites (e.g.\ \textit{nobody, nothing})  also often involve Wh-words. These expressions are typically derived by adding a negative particle to the relevant Wh-word (e.g.\ ``not who'' for `nobody' in (\ref{ex:negidf})).

\begin{exe}
  \ex\label{ex:negidf} \textsc{Murrinh-Patha (nPN: Southern Daly)}\hfill (John Mansfield, p.c.)\\
  \gll \textbf{Mere} \textbf{nangkal} nge-ma-nham.\\
  \textsc{neg} who pierce.\Rr.\Fsg.\Irr-\Appl-fear\\
  \glt `I'm not afraid of anyone.' % (2013-01-18_pb_01)
\end{exe}

A recurring analytical problem with such constructions is determining whether they form a genuine negative indefinite series in a given language or rather are existential quantifiers (i.e., plain indefinites) occurring in the scope of negation. Relevant properties that could help decide between the two options include (un)usual word order of the negation marker; for instance, in Kunbarlang, the negative particle typically immediately precedes the verb, but in clauses with such indefinites it precedes the Wh-word. Another relevant property could be the optionality/obligatoriness of the negative particle; if the negative particle may be omitted, because the verb sufficiently encodes the negative semantics, this may be considered optional negative concord rather than a series of negative indefinites.

\subsection{Temporal quantifiers}
\label{sec:tempq}
Temporal quantifiers count or measure time intervals, or more broadly, cases (i.e.\ instantiations of particular event types; \citealt{lewis75}). Examples of English temporal quantifiers include \textit{often}, \textit{sometimes}, \textit{always}, \textit{never}, and so on.  Sources on nearly half of the languages in our sample (54\ofy) contain descriptions or at least mentions of temporal quantifiers. 

In terms of their morphosyntax, the vast majority of Australian temporal quantifiers are free adverbs (\ref{ex:gniagain}) and other kinds of A-quantifiers. These A-quantifiers include clitics (Alyawarra =\textit{antiya} `still'/ `always'), verbal affixes (\ref{ex:oftsuff}), % also: bkw and garadjari
nominal affixes (\ref{ex:manytimes}), and even verbal roots ((\ref{ex:valways}); see also Pintupi (\pgcitealt{hansenhansen77}{148}) and Yugambeh (\pgcitealt{sharpe98}{171})). Although most temporal quantifiers are A-quantifiers, we notice that they are often morphologically derived from D-quantifiers. This is in line with the observations in \cite{gil93} and \cite{keenanpaperno17ov}.
\begin{exe}
  \ex\label{ex:gniagain} \textsc{Gooniyandi (nPN: Bunaban)}\hfill \pgcitep{mcgregor90}{462}\\
  \gll Nganyi nyagginboowooo \textbf{ngambiddi}-nyali.\\
  I he:will:spear:me again-\Rep\\
  \glt `I might be speared again (not necessarily by the same person).'
  % \ex\label{ex:affalws} \textsc{Garadjari (PN: Marrngu)} (\citealt[42]{sands89})\\ %% IK: it's ridiculuos how much rubbish there is. I'm so ultimately disappointed in the classical typology..\\
  % \gll Ngayidju yara gurga dja-\textbf{ngala}-gu.\\
  % I dark arise \Aux-habit-\textsc{purp}\\
  % \glt `I will always arise at night.'
  \ex\label{ex:oftsuff} \textsc{Nhirrpi (PN: Karnic)}\hfill \pgcitep{bw05}{S23}\\
  \gll Malkirri nhulu-Ru mandri-\textbf{parapara}-rla.\\
  many \Tsg.\Erg-deictic catch-often-\Prs.\Prog\\
  \glt `He often catches a lot of them.'
  \ex\label{ex:valways} \textsc{Burarra (nPN: Maningrida)}\hfill \pgcitep{green87}{87}\\
  \gll Nguburr-barmgumu-rra wupa ni-pa a-yu-ma a-\textbf{workiya}-ø.\\
  \First$>$\Second\Aug-enter-\Pc{} inside \Third\Min-s/he  \Third\Min-lie-\Ctp{} \Third\Min-do:always-\Ctp\\
  \glt `We went in to where he sleeps [lit.\ always lies].'
\end{exe}

The most frequently described temporal quantifiers have universal force (i.e., `always' (\ref{ex:valways}), `forever' (\ref{ex:4eva}), and so on). These universal force temporal quantifiers are described in 39\ofy \ languages in our sample.% 34 always's + 5 4evers
Temporal quantifiers with existential force are described in 12\ofy \  languages; these express meanings like `sometimes' (\ref{ex:smtms}), `often' (\ref{ex:oftsuff}) or `few times'. Three sources mention a negative temporal quantifier `never' (\ref{ex:never}).
\begin{exe}
  \ex\label{ex:4eva} \textsc{Yir Yoront (PN: Paman)} \hfill (\citealt[343]{alpher73})\\
  \gll n̪\'awər \textbf{monlån}$+$ar m\^a\d{l}\d{l}əl, t̪ol w\^al$+$\'aw\d{r}ən̪.\\
  that forever-\Sel{} hand-\Np-it shell that-\Sub\\
  \glt `That one will stick on forever, that spearthrower-shell.'
  \ex\label{ex:smtms} \textsc{Mawng (nPN: Iwaidjan)} \hfill (\citealt{ngaralk})\\
  \gll \textbf{Yara} k-aw-a-ø k-ampu-ma-ø mata merrk.\\
  some \Prs-\Tpl-go-\Np{} \Prs-\Tpl$>$\Third\Clveg-get-\Np{} \Clveg{} leaves\\
  \glt `Sometimes they go and get leaves.' % letter=24#e3187
  \ex\label{ex:never} \textsc{Wemba Wemba (PN: Kulin)} \hfill (\citealt[47]{hercus92})\\
  \textit{\charis wembakən} `never' $<^*$\textit{wemba} `no, not'
\end{exe}

At least four languages in our sample (Djinang, Kunbarlang, Mawng and Yir Yoront) exhibit an interesting polysemy with respect to their existential temporal quantifiers. These quantifiers are able to range either over individuals (`some') or over times (`sometimes'); in other words, they alternate between functioning as A- and D-quantifiers.  Thus, the Mawng lexical item \textit{yara} in (\ref{ex:smtms}) means `sometimes' and is a temporal adverbial, but in (\ref{ex:dyara}) means `some' and is a modifier for the nominal \textit{ja kiyap} `\Clm{} fish'.\footnote{We have confirmed with both Kunbarlang and Mawng speakers that a single expression including such an existential quantifier may be ambiguous between both A- and D-quantifier readings, i.e., (\ref{ex:smtms}) can also mean `Some of them go and get leaves.'} (We note that this polysemy does not correlate with a weak distinction between adjectives and adverbs in a language; for instance, in Kunbarlang, adjectives and adverbs are distinct categories.)

\begin{exe}
  \ex\label{ex:dyara} \textsc{Mawng (nPN: Iwaidjan)} \hfill (\citealt{ngaralk})\\
  \gll \textbf{Yara} ja kiyap k-i-w-ø.\\
  some \Clm{} fish \Prs-\Tsg\Clm-lie-\Np\\
  \glt `There is some fish.' % letter=24#e3187
\end{exe}

% I think we should cite references for all of the languages that have this. It's such an interesting property! --- IK TODO 

A notable number of languages in our sample (24/125) have a strategy to encode the meaning `\textit{n} many times'. We refer to these expressions as `times'-adverbials. We identify three major strategies that languages use to express these adverbials: (i) combining a D-quantifier with a specialized `times' affix; (ii) combining a D-quantifier with a non-specialized affix; and (iii) combining a D-quantifier with a lexical item meaning `arm' or `finger'.\footnote{We find a morphologically simple `times'-adverbial in only one language in our sample; Yir Yoront  has the lexical item \textit{thonolt} `once' (although the formative \textit{thon}- is found in other combinations).}  In the majority of languages with  these adverbials (14/24), they are built with a specialized `times'-affix.\footnote{Djabugay (\citealt{patz91}) and Djinang (\citealt{waters83}) have a free `times' lexical item for this purpose, rather than an affix.} This affix attaches to a (D-)quantifier---typically a cardinal numeral---to form a `times'-adverbial, as in (\ref{ex:manytimes}). 
\begin{exe}
  \ex\label{ex:manytimes} \textsc{Kalkatungu (PN: Kalkatungic)}\hfill \pgcitep{blake79}{152}\\
  \gll \textbf{mal̪t̪a}-\textbf{ŋujan} ŋai iŋka-n̪a pa-un̪a\\
  much-times I go-\Pst{} there-\All\\
  \glt `I went there lots of times.'
\end{exe}

The other two major patterns are also derivational. Five languages derive their `times'- adverbial from a D-quantifier by a non-specialized affix such as limitative affix or a case marker, as in  (\ref{ex:dattimes}).
\begin{exe}
  \ex\label{ex:dattimes} \textsc{Warlpiri (PN: Ngumpin-Yapa)}\hfill \pgcitep{bowler17}{969}\\
  \gll \textbf{Rdaka}-\textbf{pala}-\textbf{ku}=rna yanu Willowra-kurra.\\
  five-\Card-\Dat=\Fsg.\Sbj{} go.\Pst{} Willowra-\All\\
  \glt `I went to Willowra five times.'
\end{exe}
  
In four languages, the noun meaning `arm' (in Yir Yoront, `finger') combined with a D-quantifier yields a `times'-adverbial construction, as in (\ref{ex:2arm}).
\begin{exe}
  \ex\label{ex:2arm} \textsc{Kunbarlang (nPN: Gunwinyguan)} \hfill (Kapitonov, field notes)\\
  \gll Ka-mankang \textbf{kaburrk} \textbf{bala} \textbf{na}-\textbf{kudji} \textbf{burru}=\textbf{rnungu}.\\
  \Tsg.\Nfut-fall.\Pst{} two and \Cli-one arm=he.\Gen\\
  \glt `He fell down three times.' %sy_170525
\end{exe}

Finally, in Waluwara we find an example of the causative/verbaliser -\textit{\charis ma} attaching to the numeral \textit{\charis kutja} `two,' resulting in the verb \textit{\charis kutjama} `to do something twice' (\citealt[113]{breen71}).

\subsection{Expressing `how many'/`how much'
\label{sec:howmany}}

Over half of the languages in our sample (65\ofy) have an expression that is used to ask `how many' or `how much.' Australian languages are frequently described as having ``simple'' counting systems (e.g.\ \citealt[67]{dixon02}); for this reason, we find the prevalence  of lexicalization for `how many' to be especially noteworthy. This suggests to us that the concept of quantity and cardinality may be more salient in Australian languages than previous descriptions have proposed. Overall, we find that expressions for `how many' are features of both Pama-Nyungan and non-Pama-Nyungan languages.
%i.e., having few expressions that are used to describe the cardinalities of sets 

We identify four main strategies used by languages to express `how many': (i) having a unique lexical item for `how many'; (ii) using a Wh-word that is polysemous between quantity and other categories; (iii) morphologically deriving `how many' from another Wh-word; and (iv) expressing `how many' periphrastically. These latter three strategies are relatively uncommon, and are described in only $\sim$15/65 languages. 

Over two thirds of these 65 languages (\textit{n} = 44) have a unique lexical item used to inquire `how many' or `how much,' as in (\ref{ex:npnhm})--(\ref{ex:pnhm}). %This prevalence of lexicalization for `how many' suggests that the concept of quantity may be more salient in Australian cultures than the received wisdom has it. 
%We notice that many languages lexicalize the interrogative meaning `how many'. More precisely, just over one third of the languages in the sample (44\ofy) have a unique, non-periphrastic way to inquire about the quantity (\ref{ex:npnhm}, \ref{ex:pnhm}). 

\begin{exe}
  \ex\label{ex:npnhm}\textsc{Umbugarla (nPN: Umbugarlic)}\hfill \pgcitep{davies89}{57}\\
  \gll walalg \textbf{djugamərr} ga-rar?\\
  child how.many \Ssg-got\\
  \glt `How many kids have you got?'
  \ex\label{ex:pnhm}\textsc{Martuthunira (PN: Ngayarta)}\hfill \pgcitep{dench95}{190}\\
  \gll \textbf{Nhamintha} ngula? Kayarra jina, kayarra juwayu wirra-ngara wiyaa.\\
  how.many \Ignor{} two foot two hand boomerang-\Pl{} maybe\\
  \glt `How many were there? Maybe twenty boomerangs [lit.\ two hands and two feet of boomerangs].'
\end{exe}

A small number of languages in our sample express `how many' using a Wh-word that is polysemous with other Wh-meanings, as in MalakMalak \textit{amaneli} `what'/`how many.' (See also  \citealt[15--16]{bittnerhale95} for a discussion of Warlpiri \textit{nyajangu} `how many'/`which ones.')

\begin{exe}
  \ex\label{ex:whathm}\textsc{MalakMalak (nPN: Northern Daly)}\hfill \pgcitep{malakdict}{3}
  \begin{xlist}
    \ex\gll \textbf{amaneli} yi-de?\\
    what \Tsg.\M-go/be.\Prs\\
    \glt `What person is that? What kind of person?'
    \ex\gll dunyu-warra \textbf{amaneli} nuen-de?\\
    raintime-in what \Ssg-go/be.\Prs\\
    \glt `How old are you?' [lit.\ `How many rainy seasons have you been in?']
    
    % MB: This is interesting-- in Tryon 1977: 17, "amaneli" is glossed as "when"/"how many"!
    
 %\textit{when} (MalakMalak (nPN: Northern Daly); \citealt[17]{tryon74}),
  \end{xlist}
 \end{exe}
 
 %Further 14 languages have other ways to form such questions.  
 
 Other languages morphologically derive their expression for `how many' from other Wh-words. These source Wh-words include \textit{what} (\ref{ex:hmwhatq}), \textit{where} (\ref{howmanyex2}), and \textit{how} (Kuuku Ya'u; \citealt{thompson88}). We do not have enough tokens to make strong generalizations about morphemes that are typically used to derive `how many'; however, we have found several instances of a `quantity' suffix and two instances of Wh-word reduplication (Mara \textit{gangugangu} `how many' $>$ \textit{gangu} `what' (\citealt[174]{heath81}); Kuuku Ya'u  \textit{wantawantalu} `how many' $>$ \textit{wantantu} `how' (\citealt[91]{thompson88})).

 \begin{exe}
   \ex\label{ex:hmwhatq}\textsc{Ngiyambaa (PN: Central NSW)}\hfill \pgcitep{donaldson80}{267}\\
   \gll \textbf{minja}-\textbf{ŋalmaynj}-dji-waː=ndu giyanhdha-nha\\
   what-\textsc{quantity}-\Circ-\Excl=\Second\Nom{} fear-\Prs\\
   \glt `How many (of them) are you afraid of?'
   \ex \label{howmanyex2} \textsc{Matngele (nPN: Eastern Daly)}\hfill \pgcitep{zandvoort99}{51}
   \begin{xlist}
     \ex \gll ngun \textbf{an}-yin buy-burrayn\\
     there where-\All{} go-\Third\Aug\Sbj.go.\Impv\\
     \glt `Where's that lot going?' %(143) (p.\ 51)
     \ex \gll nida \textbf{an-buwaja} wari-mi-anyang\\
     brother where-\textsc{buwaja} have-\Impv-\Second\Min\Sbj.\Prs\\
     \glt `How many brothers do you have?'\footnote{\cite{zandvoort99} does not provide an interlinear gloss for \textit{buwaja}.} %(198) (p.\ 51)
   \end{xlist}
 \end{exe}
  %For the purpose of this chapter, we simply note that this suffix can combine with \textit{where} to express \textit{how many}.
  
 Finally, a small number of languages encode `how many'  through periphrastic constructions, as in (\ref{ex:hmprph}).
  
  \begin{exe}
  \ex\label{ex:hmprph}\textsc{Kunbarlang (nPN: Gunwinyguan)}\hfill (Kapitonov, field notes)\\
  \gll \textbf{Birlinj} \textbf{ka}-\textbf{ngunjdje} ki-karrme nakarrken?\\
  how \Tsg.\Nfut-perform.\Np{} \Ssg.\Nfut-hold.\Np{} dog\\
  \glt `How many dogs do you have?' % IK1-170525_2SY-02/27:07--10
\end{exe}

In line with the weak distinction between mass and count nouns noted in \S\ref{manymuchsection} and \S\ref{alleverysection}, we find that in a number of languages the same expression for `how many' can be used with both count nouns (to express `how many')  (\ref{ex:hmcnt}) and  mass nouns (to express `how much')  (\ref{ex:hmmass}). We have not found any descriptions of quantity Wh-terms that select specifically for count or mass nouns. 
%  One lexical item can express both `how many' and `how much': % do smth about the meat-classifier so that it doesn't look like mass noun "roo meat"
\begin{exe}
  \ex \textsc{Yanyuwa (PN: Ngarla)} \hfill (\citealt[27]{kc96})
  \begin{xlist}
    \ex \gll Li-\textbf{ngandarrangu} kal-inyamba-minmirra ambuliyalu?\\
    \Pl-how.many   they-\Refl-be.sick  before\\
    \glt `How many people were sick before [with flu like this]?' \label{ex:hmcnt}
    \ex \gll Ma-\textbf{ngandarrangu} ma-kijululu kuwu-rduma-la?\\
    \textsc{fd}-how.much  \textsc{fd}-money it.\textsc{fd}.you.\Sg-get-\Fut\\
    \glt `How much money will you get?' \label{ex:hmmass}
  \end{xlist}
  
  
  % MB: I'm swapping out the Maranunggu example for the Yanyuwa example so we can have a PN language (all but two of the examples in this section were nPN)
  % \ex\textsc{Maranunggu (nPN: Western Daly)}\hfill \pgcitep{tryon70}{72}\\
  % \begin{xlist}
  %   \ex\label{ex:hmcnt}\gll \textbf{Antyintara} awa manarrk kanatan ayi?\\
  %   how.many meat kangaroo you.see.\Nfut{} \Pst.\Aux\\
  %   \glt `How many kangaroos did you see?'
  %   \ex\label{ex:hmmass}\gll Menner \textbf{antyintara} kanara paty?\\
  %   sugar how.many you.hand.\Nfut{} have\\
  %   \glt `How much sugar have you got?'
  % \end{xlist}
  % MB: There's a great example of "how many"/"how much" in Harvey 1992, 232 for Gaagudju; I might substitute it for the Maranunggu example just because of the issue with "kangaroo meat" like you noted
  
  \ex \textsc{Gaagudju (nPN: isolate)} \hfill (\citealt[232]{harvey92})
  \begin{xlist}
    \ex \gll \textbf{yama}=\textbf{da}=\textbf{'geegirr} ga'djaalnga $\varnothing$-nee-ma\\
    {how many} turtle \Third.\Cli.\Abs-\Second.\Erg-get.\Pp\\
    \glt `How many turtles did you get?'\footnote{This Wh-term is morphologically complex. \citet[232]{harvey92} proposes that the initial component of the expression is related to the \textit{yaana-} `where' determiner, where \textit{geegirr} is the universal quantifier `all' (discussed previously in \S\ref{alleverysection}, example (\ref{univex0})).}
    \ex \gll \textbf{yama}=\textbf{da}=\textbf{'geegirr} djaarli $\varnothing$-naa-garra-y\\
    {how much} meat \Third.\Cli.\Abs-\Second.\Erg-get.\Prs\\
    \glt `How much meat do you have?' % (B718)
  \end{xlist}
\end{exe}

\subsection{Quantifier interaction}
\label{sec:qfrscope}
The interpretation of multiple quantifiers in a single expression is a classic topic in the theoretical literature on quantification (e.g.\ \citealt{szab97}). Quantifiers are described as ``scoping'' over one another; an example of such a quantifier scope interaction is as in the English sentence \textit{Some student loves every teacher}. This expression has two possible readings: (i) a single student is such that they love every teacher, and (ii) every teacher is such that they are loved by some (potentially different) student. A standard account of this ambiguity states that the existential quantifier \textit{some} can take scope above the universal quantifier \textit{every}, or vice versa. Another example of quantifier interaction is building complex quantifiers from simpler ones via, for instance, boolean compounding (as in the English expression \textit{not more than five}).

The description of these topics in Australian languages is in its infancy, with very few examples found in the literature. The majority of the relevant examples address the interpretation of quantifiers with respect to negation. Quantificational expressions, in particular the ones with existential force, are typically found to scope under negation. A small number of languages are described as having codified expressions including negation and quantifiers, e.g.\ the Mangarayi expression in (\ref{mangarayiquantneg1}), where the negated meaning is that of value judgement.

\begin{exe}
  \ex \textsc{Mangarayi (nPN: Gunwinyguan)} \hfill \pgcitep{merlan89}{37--38} \label{mangarayiquantneg1}\\
  \gll \textbf{ŋi\~{n}jag} \textbf{guyban} ga-ŋa-nidba\\
  \Proh{} little.bit \Third-\Fsg$>$\Tsg-have\\
  \glt `I have not a little bit,' i.e., `I have a lot.' %\hfill [$\neg$ $>$ $\exists$] IK: No, this is a different semantics!!! :))) MB: Ohhhhhh yeah. lol. I got overexcited with adding those. Haha.\\
\end{exe}

In Wubuy, bare common nouns scope under or above negation depending on the presence or absence of a class prefix, respectively. In (\ref{ex:nsnar}), the bare noun scopes under negation, whereas in (\ref{ex:nswide}), the noun scopes above negation. (\citet{baker08} analyses these class prefixes as topic markers, serving to indicate the scope of clausal operators.)

\begin{exe}
  \ex\label{ex:nuyscope} \textsc{Wubuy (nPN: Gunwinyguan)}\hfill \pgcitep{baker08}{145}
  \begin{xlist}
    \ex\label{ex:nsnar}\gll \textbf{waaɻi} ŋa-ŋu-kuʈaŋi $^*$\textbf{(ana-)ŋucica}\\
    nothing \Fsg.\Sarg-\Clneut-catch.\Pcon{} \phantom{$^*$(}\Clneut.\Top-fish\\
    \glt `I didn't get any fish.' \hfill [$\neg > \exists$]
    \ex\label{ex:nswide}\gll \textbf{waaɻi} ŋan-tani \textbf{ŋucica}\\
    nothing \Fsg.\Sarg$>$\Anim-spear.\Pcon{} fish\\
    \glt `I didn't spear a fish (one in particular).' \hfill [$\exists > \neg$]
  \end{xlist}
\end{exe}

Some data from Kunbarlang  suggest that word  order may also change the scopal properties of an existential. Example (\ref{ex:wlgscope}) shows that when the NP with the numeral `one' follows the negative particle, it can only scope below negation (\ref{ex:kudjinar}), but when it precedes the negative particle, there is a scope ambiguity (\ref{ex:kudjiamb}).
\begin{exe}
  \ex\label{ex:wlgscope} \textsc{Kunbarlang (nPN: Gunwinyguan)}\hfill (Kapitonov, field notes) %sy_160701\\
  \begin{xlist}
    \ex\label{ex:kudjinar} \gll \textbf{Ngunda} ki-kala \textbf{na}-\textbf{kudji} (nayi) barbung.\\
    not \Tsg.\Irr.\Pst-get.\Irr.\Pst{} \Cli-one \phantom{(}\Nm.\Cli{} fish\\
    \glt `S/he didn't get a single fish.'\hfill [$\neg > \exists$]\\
    $^*$ `S/he didn't get one fish.'\hfill $^*$[$\exists > \neg$]
    \ex\label{ex:kudjiamb} \gll Nayi \textbf{na}-\textbf{kudji} barbung \textbf{ngunda} ki-kala.\\
    \Nm.\Cli{} \Cli-one fish not \Tsg.\Irr.\Pst-get.\Irr.\Pst\\
    \glt `S/he didn't get a single fish.'\hfill [$\neg > \exists$]\\
    `S/he didn't get one fish.'\hfill [$\exists > \neg$]
  \end{xlist}
\end{exe}

%The following example (\ref{ex:notall}) from Panyjima illustrates the narrow scope of the universal quantifer with respect to negation.
%\begin{exe}
 % \ex\label{ex:notall} \textsc{Panyjima (PN: Ngayarta)}\hfill \pgcitep{dench91}{PG}\\
%  \gll \textbf{mirta} \textbf{jurlu} panu-rla pajarrangu, kutharra-kanu paja-ngarni panu-rla\\
  %not all very-\Foc{} savage two-\Lim{} wild-\Com{} very-\Foc\\
  %\glt `Not all are savage, only two are really wild.'
%\end{exe}

\pgcitet{wilkins89}{\S3.5.4} reports that in Arrernte, it is possible to combine two quantifiers within one NP (\ref{ex:mqnp}), and possibly even with scope effects. However, the data are insufficient to make any more specific conclusions.
\begin{exe}
  \ex\label{ex:mqnp} \textsc{Arrernte (PN: Arandic)} \hfill \pgcitep{wilkins89}{110}\\
  \gll kngwelye atningke ingkirreke\\
  dog many all\\
  \glt `all of the many dogs'
\end{exe}

We take these findings to indicate an uncovered richness of data to explore in the course of future fieldwork on Australian languages. However, fieldwork on these concepts is notoriously difficult and requires carefully constructed scenarios, typically with illustrations alongside, to ensure the correct understanding of the truth conditions of every example by both the consultant and the linguist.


%\subsection{Non-standard quantificational expressions}
%`and all' (found in both Panyjima and Wardaman)

%minyjarnu - ‘and all’

%\begin{exe}
%\ex \textsc{Panyjima (PN: Ngayarta)} \hfill (\citealt[65]{dench91})\\
%\gll ngatha      miyinma-rna     mantu-yu jarta\textbf{-minyjarnu}-ku  juju-ngarli-muntu-ku. \\
%1sgNOM     provide-PAST     meat-ACC    old.woman-ANDALL-ACC   old.man-PL-CONJ-ACC \\
%`I provided meat for the old women and all the old men.'
%\end{exe}

\subsection{Lexical item for `to count'}

Although it is not a quantifier itself, we are interested in the prevalence of the verb `to count' in Australian languages. In our current sample, only 9\ofy\ languages are described as having a verb `to count'. However, we suspect that the actual number of languages with this verb is %likely
somewhat higher, since only a subset of our sources include extensive wordlists, and besides this meaning might have been overlooked in the survey because of an unpredictable gloss.\footnote{Pitjantjatjara is described as having a verb \textit{kautamilani} `to count' that has been borrowed from English (\citealt[36]{goddard92}).}

%The presence of this verb suggests a familiarity with quantificational concepts like numeracy. 

\begin{exe}
  \ex \textsc{Pintupi (PN: Wati)} \hfill (\citealt[179]{pintupi77})\\
  \textit{yiltijirripungu} `to count'/`to mark the ground'\\
  (used to describe the marking of the ground with parallel marks for the purpose of counting)
  \ex \textsc{Wubuy (nPN: Gunwinyguan)} \hfill (\citealt{nuydict})
  \begin{xlist}
     \ex \textit{ngunymaa} `to examine a pile of objects'/`to count'
    \ex \textit{munduwa} `to examine closely and divide into piles'
    % \ex \textit{rangguda} `sort into diff piles of diff quality'\\
  \end{xlist}
\end{exe} 

Interestingly, several of the verbs for `to count' explicitly describe physically manipulating objects or tallies for the purpose of counting. %%This suggests to us that speakers could have repurposed existing verbs in their languages to refer to the act of counting.

\section{Conclusion and future directions}

We provided an overview of quantificational expressions in 125 Australian languages. We showed that Australian languages have equivalent expressions for all of the cross-linguistically commonly documented quantifiers, e.g.\ existential force expressions like \textit{many} and \textit{some}, and universal force expressions like \textit{all}. We showed that Australian languages typically do not lexically distinguish between quantification over mass versus count nouns, as in \S\ref{manymuchsection} on `many'/`much'. %Australian universal force quantifiers also do not lexically distinguish between quantification over singular versus plural nouns, as in \S\ref{alleverysection} on `all'/`every'.
In terms of derivational relations, Australian languages show some  robust cross-linguistic tendencies, such as formation of indefinites from interrogatives (\S\ref{sec:indefs}) and derivation of A-quantifiers from D-quantifiers (\S\ref{sec:tempq}).

In this chapter, we have focused exclusively on quantificational expressions that are commonly documented cross-linguistically and which theoretical analyses of quantification primarily address. However, while writing this chapter, we encountered a number of quantificational expressions that do not conform to our general understanding of what basic quantifiers can express. 

These lexical items can encode fairly intricate meanings related to quantification. For instance, in  {\charis{Yidiɲ}}, the nominal human suffix -\textit{ba} indicates that the individual it attaches to is one of a group of other individuals for which the predicate also holds (\ref{yidinycomplexex1}). It appears to further contribute that the relevant individual is unique in some respect when compared to other members of the group. (We find similar meanings encoded by Dyirbal \textit{-gara} `one of a pair' and {\charis{\textit{-maŋgan}}} `one of many' (\citealt[230--231]{dixon72}).)

% MB: Maybe a gloss for 'ba' could be something like 'also'? But then there's the extra "uniqueness" thing. --- IK: Yeah it's an interesting thought! I guess since that won't work as neatly for Dyirbal (and also since this is about the Unusual Quantifiers:) , it's totally fine to leave it like you did.

\begin{exe}
  \ex \textsc{\charis{Yidiɲ} (PN: Paman)} \hfill (\citealt[146]{dixon77}) \label{yidinycomplexex1}\\
  \gll yiŋu buɲaː-\textbf{ba}  gali-ŋ\\
  this woman-\textsc{ba}  go-\textsc{pres} \\
  \glt `This woman and one (or more) other people (who are not women) are going.'
  % wuguɖa-ba buɲa:-ba maɖi-:ndaŋ
  % man-ONE   woman-ONE walk-PRES
  % ‘The man and the woman---are walking uphill’ 
\end{exe}

Another case in point is \posscitet{evans95} description of affixal quantifiers in Mayali, many of which have spatial connotations, such as `dispersed' or `together in a bunch'. Again, these meanings lie outside of the familiar scope of quantificational expressions.

In conclusion, we believe that the quantificational systems of Australian languages are of significant typological and theoretical interest. It is our hope that linguists will continue to produce descriptive and theoretical work within this area. Particular topics that we believe are in need of research include the availability of scalar implicatures for existential force quantifiers (i.e., `some but not all'); the semantics of `other' expressions that are used as translational equivalents of partitive `some'; the ability of quantifiers to interact scopally and form morphosyntactically complex expressions (e.g.\ `not many'); and the prevalence and semantics of ``non-standard'' quantificational expressions (as in (\ref{yidinycomplexex1})). 

\section*{Acknowledgements}
% Vanya, I'm just putting this section in here for now because I want to make sure not to leave it out in the final one. I'm also sending this paper to our RAs, and I want them to see that we're thanking them! :P

We thank our audiences at UCLA and at ALW 2018 for their generous feedback on this project. We also thank our UCLA research assistants, Nick Curleo and Ryan Smick, for their help collecting data. %Finally, we thank Claire Bowern and Barry Alpher for 


\printbibliography
\end{document}
