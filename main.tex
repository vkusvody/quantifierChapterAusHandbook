\documentclass[12pt,egregdoesnotlikesansseriftitles]{scrartcl}
\usepackage[dvipsnames]{xcolor}
%% fonts:
\usepackage{fontspec}
\defaultfontfeatures{Ligatures=TeX,Numbers={OldStyle,Proportional}}
\setmainfont{XCharter}
\setsansfont{Alegreya Sans}
\newfontfamily\charis{CharisSIL-R.ttf}[
    BoldFont = CharisSIL-B.ttf,
    ItalicFont = CharisSIL-I.ttf,
    BoldItalicFont = CharisSIL-BI.ttf]
\usepackage{amssymb,marvosym}
\usepackage{enumerate}
\usepackage{fancyhdr}
\usepackage{authblk}
%% bibliography
\usepackage[backend=biber,
 bibstyle=%authoryear,%
biblatex-sp-unified,
 citestyle=%authoryear-ibid,%
sp-authoryear-comp,
 maxcitenames=3,
 maxbibnames=99]{biblatex}
\setlength{\bibitemsep}{0pt}
\addbibresource{mybib.bib}
\usepackage{leipzig}
\makeglossaries
% The first arg is how you type it in gloss line, e.g. {dub} means you type in \Dub; the second arg is what is printed, e.g. {dub} means small caps dub will appear in pdf; the third arg is what definition shows up in the glossary
\newleipzig{anim}{anim}{animate}
\newleipzig{aug}{aug}{augmented}
\newleipzig{card}{card}{cardinality}
\newleipzig{circ}{circ}{circumstantive}
\newleipzig{cli}{i}{class I}
\newleipzig{clii}{ii}{class II}
\newleipzig{cliii}{iii}{class III}
\newleipzig{cliv}{iv}{class IV}
\newleipzig{clm}{ma}{masculine class}
\newleipzig{clneut}{neut}{neuter class}
\newleipzig{clveg}{veg}{vegetable class}
\newleipzig{conj}{conj}{conjunction}
\newleipzig{ctp}{con}{contemporary tense}
\newleipzig{dub}{dub}{dubitative}
\newleipzig{hum}{hum}{human}
\newleipzig{ignor}{ignor}{ignorative}
\newleipzig{imm}{imm}{immediate}
\newleipzig{impv}{impv}{imperfective}
\newleipzig{indet}{indet}{indeterminate}
\newleipzig{lim}{lim}{limitative}
\newleipzig{min}{min}{minimal}
\newleipzig{nfut}{nfut}{non-future}
\newleipzig{nm}{nm}{noun marker}
\newleipzig{np}{np}{non-past}
\newleipzig{pc}{precon}{precontemporary}
\newleipzig{pcon}{pcon}{past continuous}
\newleipzig{pp}{pp}{past perfective}
%\newleipzig{proh}{proh}{prohibitive}
\newleipzig{rcgn}{rcgn}{recognitional}
\newleipzig{rempst}{rpst}{remote past}
\newleipzig{rep}{rep}{repetitive}
\newleipzig{rr}{rr}{reflexive/reciprocal}
\newleipzig{sel}{sel}{selective enclitic}
\newleipzig{sub}{sub}{subordinate marker}
\newleipzig{ua}{ua}{unit augmented}

\usepackage{multicol}
\usepackage{graphicx}
\usepackage{gb4e}
\let\eachwordone=\charis
\usepackage[colorlinks,allcolors=Blue]{hyperref}
% \usepackage[nomain]{glossaries}
% \usepackage{glossary-inline}

\newcommand{\ofy}{/125} %@ the n of languages in the survey (i.e., languages we have data for)

% total number of sources surveyed (incl. those without data) = 138

% total number of sources with data = 124

% I didn't include any instances of "personal communication" within these tallies.

\newcommand{\pn}{Pama-Nyungan}

\title{Quantification in Australian languages}
\author{Margit Bowler\thanks{\Letter:~\href{mailto:margitbowler@gmail.com}{margitbowler@gmail.com}}\authorcr UCLA \and \vspace{-.5cm}Vanya Kapitonov\thanks{\Letter:~\href{mailto:moving.alpha@gmail.com}{moving.alpha@gmail.com}}\authorcr University of Melbourne \&\ CoEDL}

\begin{document}
\maketitle

\section{Introduction}

In this chapter, we give an overview of quantificational expressions in Australian languages. We do not assume a theoretical definition of quantifiers (i.e., \citealt{heimkratzer98}); rather, we are generally concerned with lexical items that refer to quantities. This includes terms referring to vague quantities (translation equivalents to English \textit{many}, \textit{few}, \textit{several}...), properties of sets (\textit{all}, \textit{some}, \textit{no}...), cardinalities (\textit{one}, \textit{two}, \textit{three}...), Wh-words referring to quantities (\textit{how many}, \textit{how much}), indefinite pronouns (\textit{someone}, \textit{something}...), and terms referring to ``quantities'' of times, or \textit{cases}, in \posscitet{lewis75} terminology (\textit{always}, \textit{sometimes}...). We do not discuss number marking in agreement systems, non-pronominal (in)definiteness, or other lexical items that have been theoretically argued to include quantifiers in their semantic denotations, e.g.\ modals.

We show that Australian languages use a variety of morphosyntactic strategies to express quantificational concepts.

\subsection{Our data}

% 49 = number of nPN languages (including isolates)
% 71 = number of PN languages (including isolates)

Our survey is based on data from 125 Australian languages. This data is drawn from 124 published grammars, grammatical sketches, and dictionaries, as well as personal communications with some language experts. We aim to have a sample that is genetically and areally balanced as possible; we draw on data from 71 Pama-Nyungan languages from 21 subgroups and 49 non-Pama-Nyungan languages from 18 families, as well as 7 language isolates (sometimes categorized within Pama-Nyungan or non-Pama-Nyungan) and two mixed languages. This sample includes language data from all of the Australian states save Tasmania and the Australian Capital Territory.

When we report that a language ``has'' a quantifier, we mean that the sources that we consulted on the language document this quantifier. We restrain from making strong claims about languages lacking certain quantificational
expressions, since there may be gaps in the collected data (particularly in older sources; quantifier terms were not frequently collected in older language descriptions). We typically present counts as proportions of the total number of languages in our sample, i.e., for a given quantificational expression, we note how many languages have it out of a total of 125.

Finally, we generally present data as it is given in the original sources. In a small number of  instances, we standardize some interlinear glosses and orthographies.


\section{General morphosyntactic properties of  quantifier terms in Australia}

In the following sections, we give an overview of the common morphosyntactic properties of Australian quantificational expressions. We discuss the prevalence of particular quantificational expressions in \S\ref{individquantsection}.

\subsection{Lexical category of quantifier terms}

\cite{bachetal95} broadly distinguish between D(eterminer)-quantifiers and A(dverbial)-quantifiers. D-quantifiers are morphosyntactically associated with nouns, whereas A-quantifiers are (often) associated with verbs. All of the languages in our survey have D-quantifiers; quantificational concepts in these languages are expressed as nouns or as nominal modifiers (affixes, secondary predicates). 

We diagnose a given lexical item as a noun by its ability to host case marking and trigger agreement marking, as in (\ref{allerg})--(\ref{agrmarking1}).\footnote{Australian languages are often described as lacking adjectives, which are generally indistinguishable from nouns with respect to their morphosyntactic properties (\citealt[67-68]{dixon02}).}\footnote{The abbreviations used in examples are: \printglossary[style=inline,type=\leipzigtype]} Nominal quantifiers can typically stand alone as arguments, without any other associated noun, as in (\ref{agrmarking1}). Quantifiers are frequently documented in discontinuous NPs, as in (\ref{discconst}) (cf.\ \citealt[51-52]{louagieverstraete16}, who observe that quantifiers are the most frequent type of modifier to occur discontinuously in Australian languages). 

\begin{exe}
  \ex\label{allerg} \textsc{Bardi (nPN: Nyulnyulan)} (\citealt[272]{bowern12})
  \gll Nyalaboo i-ng-arr-ala-n \textbf{boonyja}-nim.\\
  there 3-\textsc{pst}-\textsc{aug}-see-\textsc{rem.pst} all-\textsc{erg}\\
  \glt `Everyone saw him.'
  \ex \textsc{Warlpiri (PN: Ngumpin-Yapa)} (\citealt[6]{bowler17})
  \gll \textbf{Panu}-ngku=lu karlaja yunkaranyi-ki.\\
  many-\textsc{erg}$=$ \textsc{3subj.pl} dig.\textsc{pst} honey.ant-\textsc{dat}\\
  \glt `Many [people] dug for honey ants.' \label{agrmarking1}
  \ex \textsc{Matngele (nPN: Eastern Daly)} (\citealt[54]{zandvoort99})
  \gll \textbf{Nembiyu} ardiminek \textbf{binya} \textbf{jawk}.\\
  one 1\textsc{ms.}do.\textsc{p} fish black.nailfish\\
  \glt `I got one black nailfish.'  \label{discconst}
\end{exe}


Many languages in our survey also have A-quantifiers; these languages can express quantificational concepts through verbal modifiers such as adverbs, as in (\ref{aquant1})--(\ref{aquant2}).


\begin{exe}
  \ex \textsc{Mayali (nPN: Gunwinyguan)} (\citealt[221]{evans95})
  \gll Gunj barri\textbf{-bebbe}-yame-ng.\\
  kangaroo 3a\textsc{p-distr}-spear-\textsc{pp}\\
  \glt `They each killed a kangaroo.'\\
  ([they]$_{key}$ [kangaroo-spear-bebbeh]$_{share}$, i.e.\ one kangaroo-killing per man) \label{aquant1}
  \ex  \textsc{Garadjari (PN: Marrngu)} (\citealt[54]{sands89})
  \gll \textbf{wiridjardu}  nga-njari-djinja.\\
  completely  eat-\textsc{cont}-3\textsc{pl.O}\\
  \glt `He ate them all up.' \label{aquant2}
\end{exe}

As far as we know, no Australian language only uses A-quantifiers; A-quantifiers always  occur in addition to D-quantifiers. Australian languages conform in this respect to \cite{bachetal95}'s generalization that while languages can lack D-quantifiers, no language can lack A-quantifiers.
%VK: nope, it's the other way round ;P

\subsection{Syntactic patterns of modification}

Quantifiers are restricted to modifying absolutive arguments in a small number of languages in our survey. To the best of our knowledge, this property is primarily described of adverbial quantifiers, as in (\ref{quantscope1})--(\ref{quantscope2}). However, \cite{harvey92} describes one nominal quantifier in Gaagudju, \textit{geegirr}, that is preferred (but not required) in combination with absolutive arguments (\ref{quantscope3}).

\begin{exe}
  \ex  \textsc{Warlpiri (PN: Ngumpin-Yapa)} (\citealt[15]{bowler17})
  \gll Wati-ngki \textbf{muku} rdilyki-pungu kurlarda-wati.\\
  man-\textsc{erg} all/completely break.\textsc{pst} spear-several\\
  \glt `The man broke all the spears.' \label{quantscope1}
  \ex \textsc{Mayali (nPN: Gunwinyguan)} (\citealt[233]{evans95})
  \gll Aban-\textbf{djangged}-bukka-ng.\\
  1$>$3\textsc{pl}-bunch-show-\textsc{pp}\\
  \glt `I showed them \textbf{the whole lot}.' \label{quantscope2}
    \ex  \textsc{Gaagudju (nPN: Isolate)} (\citealt[307]{harvey92})
  \gll ba-'rree-ng-ga=mba \textbf{geegirr} ma'rree-ya=mba \textbf{geegirr}.\\ 
  2\textsc{a-ie-fu}-take=\textsc{aug} all 1+2\textsc{a}-go.\textsc{fu=aug} all\\
  \glt `I will take all of you. We will all go.' \label{quantscope3}
\end{exe}

We note that this property does not extend to all of the quantifiers within a given language. For instance, Warlpiri has other (D-)quantifiers that are not restricted to modifying absolutive arguments.


\section{Semantic findings \label{individquantsection}}

In the following sections, we discuss the prevalence of particular quantificational expressions in the languages in our survey, and review the morphosyntactic strategies that the languages use to encode them.

\subsection{Expressing `many'/`much' \label{manymuchsection}}

The majority of the languages in our survey (109\ofy) have a lexical item that contributes a meaning like English `many,', i.e., that the cardinality of a set exceeds some contextual standard.\footnote{Only one language in our survey, the Gooniyandi mother-in-law language, is explicitly described as lacking a word for `many' (\citealt[636]{mcgregor89}).} We frequently find that languages  have more than one lexical item for `many,' as shown in (\ref{manypl1}). %(\ref{manypl2}). % in opposition to the popularized view that Australian languages have ``simple'' quantifier systems.

\begin{exe}
 \ex  \textsc{Yugambeh (PN: Ngumpin-Yapa)} (\citealt{sharpe98}) \label{manypl1}
  \begin{xlist}
    \ex \textit{kamaybu} `lots of,' `plenty,' `beyond four'
    \ex \textit{karal} `more,' `many,' `a lot,' `all,' `plenty'
    \ex \textit{walal}  `many' 
  \end{xlist} 
 % \ex  \textsc{Bardi (nPN: Nyulnyulan)} (\citealt{bowern12})  \label{manypl2}
 % \begin{xlist}
 %   \ex \textit{niimana}  `plenty,' `many'
 %   \ex \textit{ngarri} `a lot,' `much'
 %   \ex \textit{alboorr(oo)} `plenty,' `many'
 % \end{xlist} 
\end{exe}

We find that Australian languages do not lexically distinguish between quantification over count nouns versus mass nouns, i.e., the distinction between English `many' and `much.' One lexical item can therefore modify both count and mass nouns, as in (\ref{manymuch1}).

\begin{exe}
\ex \textsc{Biri (PN: Maric}) (\cite[54]{terrill98}) \label{manymuch1}
\begin{xlist} 
\ex  \gll  yara    \textbf{dhalgari}    mari        wuna-lba-dhana \\
    there    many        men-\textsc{abs}    lie-\textsc{cont-pst}-3\textsc{pl}S/A \\
    `Many men used to live here.'
\ex \gll \textbf{dhalgari}    gamu        wara-mba-li        gunhami    gamu     yinda-lbaŋa-la  \\   
    much        water-\textsc{abs}    be\textsc{-caus-pst}    that-\textsc{abs}    water-\textsc{abs}  rise-\textsc{cont-pres}-3\textsc{sg}S/A\\
    `Much rain made the river rise.'
\end{xlist}
\end{exe}

An exception to this is the use of lexical items for `big' to express a quantity meaning akin to English `much.' (At least) 12/109 languages in our sample permit their lexical item for `big' to refer to `a large quantity of [noun];' this almost always occurs in combination with mass nouns, as in (\ref{bigquant1}).\footnote{\citet[37]{louagieverstraete16} assert that in Gooniyandi, prenominal `big' functions as a quantifier, whereas postnominal `big' has an adjectival meaning:
    \vspace{-2mm}
    \begin{exe}
      \ex  \textsc{Gooniyandi (nPN: Bunuban)} (\citealt{mcgregor90})
      \begin{xlist}
      \begin{multicols}{2}
        \ex \gll \textbf{nyamani} gamba\\
        big water \\
        \glt `a lot of water'% (McGregor 1990: 260) 
        \ex \gll yoowooloo \textbf{nyamani} \\
        man big \\
        \glt `a big man' %(McGregor 1990: 265)
    \end{multicols}
      \end{xlist}
    \end{exe}
  } Only two of these languages permit `big' to combine with count nouns under a quantity reading, as in (\ref{bigquant2}).\footnote{
  In Miriwoong (nPN: Jarrakan), the lexical item \textit{ngerreguwung} `big' can undergo partial reduplication to result in \textit{ngerregungerreguwung} `a very large number'/`very many' (\citealt[43]{kofod78}).
  }$^{,}$\footnote{We find that in one language, Murrinh-Patha, `big' can be used to express universal quantificational force. It is capable of both mass and wholistic quantificational readings.
  %,i.e.\ if the head noun denotes either a mass substance or an entity with understood internal structure (whether collective (herd) or not ( the whole house))
  
\begin{exe}
\ex  \textsc{Murrinh-Patha (nPN: Southern Daly)} (John Mansfield, p.c.)
\gll Me-Ngala mup-ka \textbf{ngala} kanam-ka-wat-nime.\\
foot-big people-\textsc{top} big be.3 \textsc{sg.nfut-pauc.subj}-frequent-\textsc{pauc.m}\\
`The whole Big Foot mob come here regularly.'
\end{exe}
}

\begin{exe}
  \ex \textsc{Garrwa (nPN: Garrwan)} (\citealt{furby77}) \label{bigquant1}
  \begin{xlist}
    \ex \gll \textbf{walgu\v{r}a} wadjili \\
    big wild.honey\\
    \glt `a lot of wild honey,' `much wild honey' % p. 23
    \ex \gll \textbf{walgu\v{r}a}-nanji duŋala-nanji\\
    big-\textsc{abl} hill-\textsc{abl}\\
    \glt `from the big hill' % p. 33
\end{xlist}    
  \ex \textsc{Kalaw Kawaw Ya (PN: Western Torres Strait)} (\citealt[141]{fo91}) \label{bigquant2}
  \gll Yan burumiya lumiz +war moebaygan nanga burum        \textbf{koeyma}    mathan.\\
  in.vain pig.\textsc{com} hunt.\textsc{pr.pf} other person.\textsc{erg} when   pig.\textsc{abs}    big.\textsc{adv}    kill.\textsc{pr.pf}\\
  \glt `He hunted in vain for a pig while the others bagged many.'
  
  % MB: Vanya, the more I look at this example, the more I think it's not a good counterexample. The word "koeyma" is glossed as "big.ADV," suggesting that it's actually an adverbial modifying the verb.
\end{exe}

We note in \S\ref{alleverysection} that a small number of languages in our survey have lexical items that can be interpreted as either `many'/`much' or `all'/`every', i.e.\ they are compatible with both existential and universal force.

\subsection{Expressing `all'/`every' \label{alleverysection}}

Approximately half of the languages in our survey (63\ofy) have at least one strategy for expressing universal quantification over individuals. Like in \S\ref{manymuchsection}, we find that Australian languages do not lexically distinguish between universal quantification over count nouns versus mass nouns, as in (\ref{univmasscount1}). (We take \textit{gaarra} `salt water' in (\ref{univmasscount3}) to be a mass noun.)

\begin{exe}
\ex  \textsc{Bardi (nPN: Nyulnyulan)} (\citealt{bowern12}) \label{univmasscount1}
\begin{xlist}
\ex %\textbf{Boonyja}gid ambooriny \textbf{boonyja} lagallagal ingarrganyinan barda.\\
\gll \textbf{Boonyja}=gid ambooriny \textbf{boonyja} lagal-lagal   i-nga-rr-ganyi-n-an barda.\\
all=\textsc{then} people all climb-\textsc{redup}  3-\textsc{pst-aug}-climb-\textsc{cont-rem.pst} away\\
`Then all the people were climbing up [to get away from the rising water].'\label{univmasscount2}

%Inyjoordina gaarra \textbf{boonyja}.\\
\ex \gll i-ny-joordi-na gaarra \textbf{boonyja}.\\
3\textsc{m-pst}-dry.up-\textsc{rem.pst} salt.water all\\
`The sea all dried up.'\label{univmasscount3} %(p.\ 710) 
\end{xlist}
\end{exe}


Furthermore,  Australian universal quantifiers do not lexically distinguish between quantification  over subparts of a singular count noun versus quantification over sets of individuals or mass nouns, as in (\ref{univex0}). (We take \textit{walaalu} `country' in (\ref{univex1}) to be a singular count noun.)

% MB: Vanya, if you think this isn't a good enough example of a singular count noun (I know, it's not great), we can switch to the old stingray example. I only took that one out because it was the same Bardi quantifier as in the above examples. --- IK: I think this one is swell! (I took some formatting liberties whilst looking at it, hope you don't mind)

\begin{exe}
 \ex \textsc{Gaagudju (nPN: Isolate)} (\citealt[307]{harvey92}) \label{univex0}
  \begin{xlist}
      \ex \gll walaalu $\varnothing$-naana \textbf{geegirr}.\\
    country \Cliv-burn.\Pp{} all\\
    \glt `The country is all burnt.' \label{univex1}
    \ex \gll djirriingi njinggooduwa yaa-bu=mba \textbf{geegirr}.\\
    man woman \Third\Cli-went=\Aug{} all\\
    \glt `The men and women have all gone.' \label{univex2}
  \end{xlist}
\end{exe}

Australian universal quantifiers also do not lexically distinguish between quantification over singular versus plural nouns, i.e.\ the distinction between English `every' and `all.' This falls out in part from the fact that person and number agreement marking and plural nominal marking tend to be optional in some contexts in Australian languages (). As a result, it can be difficult to diagnose whether a given quantifier is modifying a singular or a  plural noun. %We show a (lack of) such a contrast in (\ref{}). 


We describe four primary strategies for expressing universal quantification over individuals, in the order of their frequency. The most common strategy is having a unique lexical item with universal force;  x/y languages in our sample use this strategy. We give examples of quantifiers with strictly universal force in (\ref{univex3})--(\ref{univex4}).

\begin{exe}
  \ex  \textsc{Arrernte (PN: Arandic)} (\citealt[132]{wilkins89})
  \gll Alertekwenhe pmere \textbf{ingkirreke} artwe-kenhe, artwe-kenhe pmere.\\
  there place all man-\textsc{poss} man-\textsc{poss} place\\
  \glt `That there (pointing to a particular site) was a place for all men, a men's site.' \label{univex3}
  \ex  \textsc{Garadjari (PN: Marrngu)} (\citealt[48]{sands89}) 
  \gll \textbf{Djarin}-dja barda-ngka yilba-gu-djinja.\\
  every-\textsc{loc} sun-\textsc{loc}   throw-\textsc{fut}-3.\textsc{pl}\\
  \glt `Every day he threw them [the people].' \label{univex4}
\end{exe}

Two other strategies for expressing universal quantification  involve the lexical item for `many.' A small number of languages () have quantifiers that are ambiguous between existential and universal force, as in (\ref{manyallambig1}). A single lexical item can therefore be interpreted as `many' or `all'/`every,' depending on the context. At present, we speculate that these lexical items have an underlying meaning of `many' that can be pragmatically strengthened to `all'/`every' in some contexts; however, much further fieldwork is needed to determine how and when this strengthening occurs.

\begin{exe}
  \ex  \textsc{Gugada (PN: Thura-Yura)} (\citealt[56,65]{platt72}) \label{manyallambig1}
  \begin{xlist}
    \ex \gll badu ŋurbara \textbf{muɻga} {djiɳɖu galaɭa} njina:djinj. \\ 
    man  strange  many/all  midday         sit.down\\
    \glt `A lot of strangers sat down at midday'
    \ex \gll uɭa ambuɖa \textbf{muɻga} ŋur-ŋga  \\
    boy   small    many/all      camp-\textsc{loc}\\
    \glt        `All the boys are at camp.'%\footnote{\cite{platt72} reports that \textit{muɻga} historically only had the meaning `many,' which developed into `all.'}
  \end{xlist}
 \end{exe}
 
 A still smaller number of languages morphologically derive their universal force quantifier from `many'; we observe this in only x/y languages in our sample.\footnote{We find evidence for the opposite pattern in one language, Yir Yoront. In this language, reduplicating the monomorphemic lexical item \textit{moqo} `all' yields the (existential force) value judgment quantifier `quite a few':

\begin{exe}
  \ex \textsc{Yir Yoront (PN: Paman)} (\citealt[375]{alpher73})\\
  \textit{\charis moqmor} `quite a few' $<^*$\textit{\charis moqo} `all'
\end{exe}}  Languages accomplish this  through a number of morphological strategies. These include (i) the addition of a lexical item meaning `only' or `still'; (ii) partial or total reduplication of `many'; 


\begin{exe}
  \ex  \textsc{Matngele (nPN: Eastern Daly)} (\citealt{zandvoort99})
  \begin{xlist}
    \ex \gll woerreng \textbf{mutjurr} lerr-ma-burrudak-awa\\
    mosquito many bite-\textsc{impf}-3AS.stand\textsc{p}-1\textsc{mo}\\
    \glt `Lots of mosquitoes were biting me.' (ex.~353) %(p.\ 54)
    \ex \gll mi ngarru-ma-errerr \textbf{mutjurr-ayu}-rnung\\
    tucker 1\textsc{aug-prm}-1\textsc{ncl} many-only-\textsc{purp}\\
    \glt `This tucker belongs to all of us.' (ex.~305)
    \end{xlist}
\end{exe}


 
 

\subsection{Expressing `several'/`few'}

\subsection{Expressing the partitive `some'}
\label{sec:some}
In this section we confine our attention to the partitive `some', i.e.\ the one that denotes a proportion, rather than the existential/indefinite marker `some' (which in English, for instance, can be phonologically reduced to [{\charis sm̩}], unlike the partitive one). Descriptions of 31 language in our sample report a strategy to express the meaning `some (of the \textit{n}s)'.\footnote{In case the description did not comment on the semantics of the item in question, it was often not possible to tell with certainty which `some' we were dealing with and the decision was hard to make in many such cases. Given that the grammatical marking of definiteness is not a feature frequrently found in Australian languages, we would normally give the borderline cases the benefit of the doubt and count them as partitives. We would also like to underscore that we are not suggesting an Anglocentric point of view where a lexical item is necessarily at risk of such ambiguity; rather, our concern stems from the fact that our sources are all in English, and wherever the mere translation/gloss is all information there is, the uncertainty arises due to the ambiguity in English.} %@ a reference is needed for the definiteness claim in the fn.
Among these languages, at least 14 have polysemy of `some' and `other', i.e.\ use the lexical item meaning `(an)other' (or `different') to express the partitive `some' (\ref{ex:smotherbur}--\ref{ex:smothernyan}). Over a half languages (n = 18) use a dedicated lexical item for the partitive `some' (\ref{ex:specsmawa}--\ref{ex:specsmwlg}).
\begin{exe}
  \ex\label{ex:smotherbur} \textsc{Burarra (nPN: Maningrida)}\hfill \pgcitep{green87}{84}\\
  \gll an-\textbf{nerranga} an-mola  rrapa  an-\textbf{nerranga}  an-bachirra.\\
  \Third.\Min-other \Third\Min-good and \Third\Min-other \Third\Min-wild\\
  \glt `Some are friendly and some are the angry kind.'
  \ex\label{ex:smothernyan} \textsc{Nyangumarta (PN: Marngu)}\hfill \pgcitep{sharp04}{258}\\
  \gll mungka wupartu mayi-rrangu kurrngal \textbf{jinta} juri \textbf{jinta} kari.\\
  tree  small vegetable.food-\Pl{} many other sweet other bitter\\
  \glt `The small tree/bush has lots of fruit (pilirta), some are sweet and some are sour.'
\end{exe}

\begin{exe}
  \ex\label{ex:specsmawa} \textsc{Awabakal (PN: Yuin-Kuri)}\hfill \pgcitep{lissarrague06}{ex.~177}\\
  \gll Anti=pu \textbf{winta} kuri.\\
  here=\Excl{} some.\Abs{} men.\Abs\\
  \glt `Some of the men are here.' % the pdf is quite fucked up and the hard copy seems missing from the melbourne lib, but I thought it'd be good to have this one language for some diversity.
  \ex\label{ex:specsmwlg} \textsc{Kunbarlang (nPN: Gunwinyguan)}\hfill \citep{ikthesis}\\
  \gll Ngurnda ki-kala ngob nayi barbung la \textbf{na}-\textbf{yika} ka-(rnak)-kalng.\\
  not \Tsg.\Irr.\Pst-get.\Irr.\Pst{} all \Nm.\Cli{} fish \Conj{} \Cli-some \Tsg.\Nfut-\Lim-get.\Pst\\
  \glt `S/he didn't get all the fish, but only got some.' % sy_160701
\end{exe}

The scalar implicature `some but not all', which is the most reliable characteristic of the partitive `some' (as opposed to the indefinite `some'), is a fine semantic judgement that requires careful testing in context. Unfortunately, such information is largely unavailable in the presently available sources, with the scalar implicature confirmed explicitly only for a handful of languages. % looks like pitjantjatjara, umpila, murrinhpatha, kunbarlang, and judging from exx probably also burarra, nyangumarta, maybe awabakal

% IK: So there are a few cells where there is a "no" for the L having a `some' expression, but it just looks to me like some Nick bullshit. I.e. he didn't see one and concluded there wasn't one. That's clearly so for the two Giimbiyu Ls (and then for the third one, mengerrdji, he says "yes", apparently b/c there's a little table that has a word for `other' !!!), and I believe I checked the garadjari and umbugarla as well.

\subsection{Indefinite pronouns}
\label{sec:indefs}
Lexical items for expression of indefinites (such as \textit{someone}, \textit{nothing}, \textit{anywhere}, \textit{whoever} etc.) and ignoratives (\textit{whatchamacallit}) are reported for 43 languages in the sample. In the vast majority of cases the indefinite pronouns are based on the interrogatives. Such are 36 languages, which further fall into two groups:
\begin{enumerate}[(i)]
\item those where indefinite pronouns are identical in form to \textit{wh}-words (25 languages), i.e.\ one form is ambiguous between interrogative and indefinite readings (\ref{ex:blnidf})
\item those where there is a morphological operation (optional or obligatory) deriving indefinites from \textit{wh}-words (at least 11 languages, see below)
\end{enumerate}
\begin{exe}
  % \ex\label{ex:gniidf} \textsc{Gooniyandi (nPN: Bunaban)} (\citealt[147]{mcgregor90})\\
  % \gll \textbf{ngoonyi}-yidda wardginggiri.\\
  % which-\All{} you.go\\
  % \glt (1)  `Where are you going?' \\
  % (2) `You're going somewhere.'
  \ex\label{ex:blnidf} \textsc{Bilinarra (PN: Ngumpin-Yapa)}\hfill \pgcitep{nordlinger90}{37}\\
  \gll \textbf{ngantu}-rlu-nga pa-ni.\\
  who-\Erg-\Dub{} hit-\Pst{}\\
  \glt (1) `Who hit him?'\\
  (2) `Maybe someone hit him.'
\end{exe}

Oftentimes the identical forms will have different distributional tendencies, e.g.\ when used as interrogatives these pronouns will appear clause-initially, but enjoy more freedom of placement when used as indefinites (\ref{ex:bkwidf}).
\begin{exe}
  \ex\label{ex:bkwidf} \textsc{Bininj Kun-wok (nPN: Gunwinyguan)}\hfill \pgcitep{evans03}{280--1}\\
  \begin{xlist}
    \ex\gll \textbf{Njale} bene-boken kabene-h-na-n?\\
    what \Third.\Ua-two \Third.\Ua-\Imm-see-\Np\\
    `What are they two looking at?' % ex 7.79a
    \ex\gll bu \textbf{njale} ngarri-ma-ng\ldots\\
    \Sub{} what \First.\Aug-get-\Np\\
    `and if we get something\ldots' % ex 7.90
  \end{xlist}
\end{exe}

In the languages that can or must differentiate the interrogatives and indefinites morphologically, the latter are typically derived via an indefinite, ignorative, or dubitative affix/particle (\ref{ex:idftime}), or via reduplication (\ref{ex:idfrdp}).\footnote{Interestingly, Bardi (nPN: Nyulnyulan) can express the indefinite pronoun `something' using a compound of `who' and `nose': \textit{angginimal} `something' [lit.\ \textit{anggaba} `who' + \textit{niimal} `nose'] (\citealt[321]{bowern12}).}
\begin{exe}
  \ex\label{ex:idftime} \textsc{Djambarrpuyŋu (PN: Yolŋu)}\hfill \pgcitep{wilkinson91}{393}\\
  \gll %nhämunha/nhämuny - ‘How many’ (stem)
  Ga djäma nhe dhu ga-a yindi nhe dhu ga djäma \textbf{ŋula} \textbf{nhämunha} dhuŋgarra ŋurraka$+$m\\
  and work you \Fut{} \Impv-\First{} big you \Fut{} \Impv-\First{} work \Indf2 how.many year throw$+$\First\\
  \glt `And you are working, you are working (on something) big, lasting for an indefinite amount of time.'
  \ex\label{ex:idfrdp} \textsc{Arabana-Wangkangurru (PN: Karnic)}\hfill \pgcitep{hercus94}{129}\\
  \gll \textbf{Thiyara}\char`~\textbf{thiyara} yuka-ka \textbf{minha}\char`~\textbf{minha} mapi-rnda, partyarna ngawi-lhiku waya-rnda.\\
  \textsc{rdp}\char`~which.way go-\Pst{} \textsc{rdp}\char`~what do-\Prs{} all hear-\Purp{} wish-\Prs\\
  \glt `Wherever he went and whatever he did, I want to hear it all.' %(p.\ 129)
\end{exe}
  % \ex  \textsc{Ngiyambaa (PN: Central NSW)} (\citealt[271]{donaldson80}) \\
  % \gll \textbf{ŋa:ndi-ŋa:ndi-ga:} manabi-nji.\\
  % who\char`~who+ \textsc{abs}-\textsc{ignor} hunt-\textsc{past}\\
  % \glt `Whoever went hunting, I don't know.' 
  % \ex  \textsc{Tiwi (Isolate)} (\citealt[57]{osborne74})
  % \begin{xlist}
  %   \ex \charis{\textbf{kuwani} jilkəɹimi?}\\
  %   `Who did it?'
  %   \ex \textbf{aramu-kuwarni}.\\
  %   `Someone or other.'
  % \end{xlist}

The negative indefinites are often formed from the \textit{wh}-words via adding a negative particle (i.e.\ ``not who'' for `nobody'; see (\ref{ex:negidf})). A recurrent analytical problem with such constructions is whether they form a genuine negative indefinite series in a given language or rather are existential quantifiers (i.e.\ plain indefinites) in scope of negation. %%Possible arguments in favour of this grammaticalising into a dedicated indefinite series
Relevant properties that can help decide include e.g.\ (un)usual word order of negation marker (as in Kunbarlang, where the negative particle immediately precedes the verb, but in clauses with such indefinites it precedes the interrogative/indefinite) or (un)availability of alternative constructions (e.g.\ the negation only expressed in the verb).
\begin{exe}
  \ex\label{ex:negidf} \textsc{Murrinh-Patha (nPN: Southern Daly)}\hfill (John Mansfield, p.c.)\\
  \gll \textbf{Mere} \textbf{nangkal} nge-ma-nham.\\
  \textsc{neg} who pierce.\Rr.\Fsg.\Irr-\Appl-fear\\
  \glt `I'm not afraid of anyone.' % (2013-01-18_pb_01)
\end{exe}

Besides this major strategy of forming indefinites from interrogatives, there are two minor trends. One is the employment of generic nouns or classifiers (\ref{indetclass1}) to fulfil the indefinite pronoun function, found in 6 languages in our sample. The other option, of course, is for the language to have dedicated lexical items for indefinites and/or ignoratives. We have identified 8 languages that have such items.
\begin{exe}
  \ex\label{indetclass1} \textsc{Burarra (nPN: Maningrida)}\hfill \pgcitep{green87}{9}\\
  \gll an=gata    \textbf{ana}=\textbf{nga}            joborr    gu-rrumu-rra  abu-bu-na            aburr-workiya-na.\\
  \Third\Min=that.\Rcgn{}   \Third\Min\Hum=\Indet{}   law     \Third\Min$>$\Third\Min-break-\Pc{}  \Third\Aug$>$\Third\Min-hit-\Pc{}    \Third\Aug-do:always-\Pc\\
  \glt %`Whoever broke the law they hit him all the  time.'\\ %% Margaret Carew's comment: "I would stay with Rebecca’s original translation, the clause with rruma ‘break’ is subordinate. An-gata is a recognitional demonstrative."  %IK: now, I don't really know ehich one she took as the original, but I reckon she meant to keep this one:
  `When someone breaks the law, they always hit him.' [translation ours---MB\&IK]
  \ex \textsc{Kalkatungu (PN: Kalkatungic)}\hfill \pgcitep{blake79}{104--5}
  \begin{xlist}
    \ex \textit{\charis n̪ani} `who'; \textit{\charis n̪aka} `what'
    \ex ``The interrogatives are not used as indefinites\ldots\ \textit{\charis ŋarpa} is the indefinite `some creature'\ldots\ \textit{\charis min̪aŋara} is `something'\thinspace''
  \end{xlist}
\end{exe}

Although indefinite pronouns, often directly associated with existential and universal quantifiers in semantic analyses, are a prime tool for the study of scope ambiguities and quantifier interaction, we do not find discussion of these issues in the currently available descriptions. %@ and add something else here to sum it up.
%@ IK: also, I haven't really talked about their semantics much --- I guess I'll need to expand

\subsection{Temporal quantifiers}
\label{sec:tempq}
Descriptions for nearly a half of the languages in the sample (54\ofy) contain descriptions or at least mentions of temporal quantifiers. These are the expressions that count or measure time intervals, or more broadly, cases (i.e.\ instantiations of particular event types), viz.\ `often', `sometimes', `always', `never', etc. In terms of their morphosyntax, the absolute majority of temporals are free adverbs (\ref{ex:gniagain}) and other kinds of A-quantifiers: clitics (e.g., Alyawarra =\textit{antiya} `still; always'), verbal affixes (e.g., Nhirrpi -\textit{parapara} `often'; \ref{ex:oftsuff}), % also: bkw and garadjari
nominal affixes (e.g., Kalkatungu -\textit{ŋujan} `(x many) times'; see \ref{ex:manytimes} below), and even verbal roots (in Burarra (\ref{ex:valways}; also \ref{indetclass1}), Pintupi and Yugambeh). Although most TQs are A-quantifiers, we notice that they are often derived from D-quantifiers. This is in line with the observations in \citealt{gil93,keenanpaperno17ov}.
%@ discuss wbp D-temporal with Margit
\begin{exe}
  \ex\label{ex:gniagain} \textsc{Gooniyandi (nPN: Bunaban)} (\citealt[462]{mcgregor90})\\
  \gll Nganyi nyagginboowooo \textbf{ngambiddi}-nyali.\\
  I he:will:spear:me again-\Rep\\
  \glt `I might be speared again (not necessarily by the same person).'
  % \ex\label{ex:affalws} \textsc{Garadjari (PN: Marrngu)} (\citealt[42]{sands89})\\ %% IK: it's ridiculuos how much rubbish there is. I'm so ultimately disappointed in the classical typology...
  % \gll Ngayidju yara gurga dja-\textbf{ngala}-gu.\\
  % I dark arise \Aux-habit-\textsc{purp}\\
  % \glt `I will always arise at night.'
  \ex\label{ex:oftsuff} \textsc{Nhirrpi (PN: Karnic)}\hfill \pgcitep{bw05}{S23}\\
  \gll Malkirri nhulu-Ru mandri-\textbf{parapara}-rla.\\
  many \Tsg.\Erg-deictic catch-often-\Prs.\Prog\\
  `He often catches a lot of them.'
  \ex\label{ex:valways} \textsc{Burarra (nPN: Maningrida)}\hfill \pgcitep{green87}{87}\\
  \gll Nguburr-barmgumu-rra wupa ni-pa a-yu-ma a-\textbf{workiya}-ø.\\
  \First$>$\Second\Aug-enter-\Pc{} inside \Third\Min-s/he  \Third\Min-lie-\Ctp{} \Third\Min-do:always-\Ctp\\
  \glt `We went in to where he sleeps [lit.\ `always lies'].'
\end{exe}

A noticable number of languages (n = 24) have a way to encode the meaning `\textit{n} many times', which we call `times'-adverbials. In the majority of cases (n = 14) these are built with a `times'-affix (Djabugay and Djinang have a free particle, rather than an affix). Such an affix attaches to a (D-)quantifier---often a cardinal numeral,---forming a `times'-adverbial (\ref{ex:manytimes}). The other two patterns are also derivational. Five languages derive the `times'-adverbial from a D-quantifier by a non-specialised affix, such as limitative or a case marker (\ref{ex:dattimes}). Still in other four languages %a construction with
the noun meaning `arm' (`finger' in Yir Yoront) combined with a quantifier % is used
yields a `times'-adverbial construction (\ref{ex:2arm}).
\begin{exe}
  \ex\label{ex:manytimes} \textsc{Kalkatungu (PN: Kalkatungic)}\hfill \pgcitep{blake79}{152}\\
  \gll \textbf{mal̪t̪a}-\textbf{ŋujan} ŋai iŋka-n̪a pa-un̪a\\
  much-times I go-\Pst{} there-\All\\
  \glt `I went there lots of times.'
  \ex\label{ex:dattimes} \textsc{Warlpiri (PN: Ngumpin-Yapa)}\hfill \pgcitep{bowler17}{969}\\
  \gll \textbf{Rdaka}-\textbf{pala}-\textbf{ku}=rna yanu Willowra-kurra.\\
  five-\Card-\Dat=\Fsg.\Sbj{} go.\Pst{} Willowra-\All\\
  \glt `I went to Willowra five times.'
  \ex\label{ex:2arm} \textsc{Kunbarlang (nPN: Gunwinyguan)} (\citealt{ikthesis}~to~app.)\\
  \gll Ka-mankang \textbf{kaburrk} \textbf{bala} \textbf{na}-\textbf{kudji} \textbf{burru}=\textbf{rnungu}.\\
  \Tsg.\Nfut-fall.\Pst{} two and \Cli-one arm=he.\Gen\\
  \glt `S/he fell down three times.' %sy_170525
\end{exe}

Finally, Yir Yoront has a simplex lexical item \textit{tonolt} meaning `once', and Waluwara there is an example of the causative/verbaliser -\textit{\charis ma} attaching to the numeral \textit{\charis kutja} `two', resulting in the verb \textit{\charis kutjama} `to do something twice'. %breen71:113

In terms of semantics of the temporal quantifiers, most often reported are the universal strength ones (i.e., `always' (\ref{ex:valways}), `forever' (\ref{ex:4eva}), et seq.; n = 39). % 34 always's + 5 4evers
Existentials are described in 12 languages, including meanings like `sometimes' (\ref{ex:smtms}), `often' (\ref{ex:oftsuff}) or `few times'. Three sources mention a negative temporal quantifier `never' (\ref{ex:never}).
\begin{exe}
  \ex\label{ex:4eva} \textsc{Yir Yoront (PN: Paman)} (\citealt[343]{alpher73})\\
  \gll n̪\'awər \textbf{monlån}$+$a\d{r} m\^a\d{l}\d{l}əl, t̪ol w\^al$+$\'aw\d{r}ən̪.\\
  that forever-\Sel{} hand-\Np-it shell that-\Sub\\
  \glt `That one will stick on forever, that spearthrower-shell.'
  \ex\label{ex:smtms} \textsc{Mawng (nPN: Iwaidjan)} (\citealt{ngaralk})\\
  \gll \textbf{Yara} k-aw-a-ø k-ampu-ma-ø mata merrk.\\
  some \Prs-\Tpl-go-\Np{} \Prs-\Tpl$>$\Third\Clveg-get-\Np{} \Clveg{} leaves\\
  \glt `Sometimes they go and get leaves.' % letter=24#e3187
  \ex\label{ex:never} \textsc{Wemba Wemba (PN: Kulin)} (\citealt[47]{hercus92})\\
  \textit{\charis wembakən} `never' $<^*$\textit{wemba} `no, not'
\end{exe}

At least four languages in our sample exhibit an interesting polysemy of existential quantifiers, which are able to range either over individuals (`some') or over times (`sometimes'); in other words, they alternate between A- and D-quantifier morphosyntax. These languages are Djinang, Kunbarlang, Mawng and Yir Yoront. This polysemy does not correlate with weak distinction between adjectives and adverbs in a language---for instance, in Kunbarlang they are distinct categories. Thus, the Mawng word \textit{yara} in (\ref{ex:smtms}) means `sometimes' and is a temporal adverbial, but in (\ref{ex:dyara}) means `some' and is a modifier for the nominal \textit{ja kiyap} `\Clm{} fish':
\begin{exe}
  \ex\label{ex:dyara} \textsc{Mawng (nPN: Iwaidjan)} (\citealt{ngaralk})\\
  \gll \textbf{Yara} ja kiyap k-i-w-ø.\\
  some \Clm{} fish \Prs-\Tsg\Clm-lie-\Np\\
  \glt `There is some fish.' % letter=24#e3187
\end{exe}


\subsection{`How many'/`how much'}

One lexical item can express both `how many' and `how much':

\begin{exe}
  \ex  \textsc{Maranunggu (nPN: Western Daly)} (\citealt[72]{tryon70})
  \begin{xlist}
    \ex \gll \textbf{Antyintara}  awa manarrk    kanatan                 ayi?\\
    how.many meat kangaroo you.see.\textsc{nonfut}  \textsc{past.aux}\\
    `How many kangaroos did you see?'
    \ex \gll Menner \textbf{antyintara} kanara paty?\\
    sugar how.much you.hand.\textsc{nonfut} have\\
    `How much sugar have you got?'
  \end{xlist}
\end{exe}

\printbibliography
\end{document}