% The first arg is how you type it in gloss line, e.g. {dub} means you type in \Dub; the second arg is what is printed, e.g. {dub} means small caps dub will appear in pdf; the third arg is what definition shows up in the glossary
\newleipzig{anim}{anim}{animate}
\newleipzig{aug}{aug}{augmented}
\newleipzig{card}{card}{cardinality}
\newleipzig{circ}{circ}{circumstantive}
\newleipzig{cli}{i}{class I}
\newleipzig{clii}{ii}{class II}
\newleipzig{cliii}{iii}{class III}
\newleipzig{cliv}{iv}{class IV}
\newleipzig{clm}{ma}{masculine class}
\newleipzig{clneut}{neut}{neuter class}
\newleipzig{clveg}{veg}{vegetable class}
\newleipzig{conj}{conj}{conjunction}
\newleipzig{cont}{cont}{continuous}
\newleipzig{ctp}{con}{contemporary tense}
\newleipzig{dub}{dub}{dubitative}
\newleipzig{hab}{hab}{habitual}
\newleipzig{hum}{hum}{human}
\newleipzig{ignor}{ignor}{ignorative}
\newleipzig{imm}{imm}{immediate}
\newleipzig{impf}{impf}{imperfective}
\newleipzig{impv}{impv}{imperfective}
\newleipzig{indet}{indet}{indeterminate}
\newleipzig{lim}{lim}{limitative}
\newleipzig{min}{min}{minimal}
\newleipzig{nfut}{nfut}{non-future}
\newleipzig{nm}{det}{determiner}
\newleipzig{np}{np}{non-past}
\newleipzig{pauc}{pauc}{paucal}
\newleipzig{pc}{precon}{precontemporary}
\newleipzig{pcon}{pcon}{past continuous}
\newleipzig{pp}{pp}{past perfective}
\newleipzig{priv}{priv}{privative}
\newleipzig{prm}{prm}{prominence marker}
\newleipzig{rcgn}{rcgn}{recognitional}
\newleipzig{rdp}{rdp}{reduplication}
\newleipzig{refr}{refr}{referential}
\newleipzig{rempst}{rpst}{remote past}
\newleipzig{rep}{rep}{repetitive}
\newleipzig{rr}{rr}{reflexive/reciprocal}
\newleipzig{sel}{sel}{selective enclitic}
\newleipzig{sub}{sub}{subordinate marker}
\newleipzig{ua}{ua}{unit augmented}