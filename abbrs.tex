% The first arg is how you type it in gloss line, e.g. {dub} means you type \Dub; the second arg is what is printed, e.g. {dub} means small caps dub will appear in pdf; the third arg is what definition shows up in the glossary
\newleipzig{appl}{appl}{applicative}
\newleipzig{aug}{aug}{augmented}
\newleipzig{dub}{dub}{dubitative}
\newleipzig{hum}{hum}{human}
\newleipzig{imm}{imm}{immediate}
\newleipzig{impv}{impv}{imperfective}
\newleipzig{indet}{indet}{indeterminate}
\newleipzig{min}{min}{minimal}
\newleipzig{np}{np}{non-past}
\newleipzig{pc}{precon}{precontemporary}
\newleipzig{rcgn}{rcgn}{recognitional}
\newleipzig{rr}{rr}{reflexive/reciprocal}
\newleipzig{sub}{sub}{subordinate marker}
\newleipzig{ua}{ua}{unit augmented}